\documentclass[a4paper,12pt]{article}
\usepackage[utf8]{inputenc}
\usepackage{graphicx}
\usepackage{amssymb}
\usepackage{amsmath}
\usepackage[T2A]{fontenc}
\usepackage[russian]{babel}
\pagenumbering{gobble}
\usepackage{wrapfig}


\begin{document}

\title{<<Дискретная математика: \\ комбинаторика и вероятность>> \\ \vspace{12pt} Домашнее задание}
\author{Байдаков Илья}
\date{\today}
\maketitle


\section*{Задание 1}
\centerline{\includegraphics[scale=1.1]{img/digraphHW.pdf}}
Граф $G$ изображен на рисунке выше.\\
{\bf a)} Найдите максимальную длину простого цикла в графе $G$. Укажите
все различные простые циклы максимальной длины. (Достаточно предъявить ответ)\\
{\bf б)} Найдите компоненты сильной связности графа $G$. (Достаточно предъявить ответ)\\
{\bf в)} Какое минимальное число рёбер необходимо добавить в граф $G$, чтобы
он стал сильно связным? (Необходимо обоснование ответа)\par
\newpage
\noindent {\bf Решение.} \\
{\bf а)} Максимальная длина простого цикла в графе $G$ равна трём. Это циклы $ADCA$ и $ABCA$.\\
{\bf б)} Компоненты сильной связности: $ABC$, $ADGC$, $E$, $F$.\\
{\bf в)} Одно ребро достаточно добавить. \\
{\bfОбоснование.} Рассмотрим ребро $FA$. Оно соединяет компоненты сильной связности $F$ и $ADGC$, $F$ и $ABC$. Из $ADGC$ достижима $E$, а из неё~$F$. Значит, все ранее описанные компоненты сильной связности попарно достижимы и новый граф будет связным.

\section*{Задание 2}
В некоторой компании $7$ рабочих групп $a,b,c,d,e,f$ и $g$. В пятницу необходимо провести собрания в каждой рабочей группе по отдельности, причем каждое собрание можно планировать в один из $4$ временных слотов: \\
\centerline{9:00 - 10:45,\qquad 11:00 - 12:45,\qquad 13:15 - 15:00\quad и\quad  15:15 - 17:00.}

\noindent Кроме того, некоторые сотрудники участвуют сразу в нескольких группах:
\begin{itemize}
\item есть те, кто одновременно состоят в $a,b,c$ и $d$;

\item несколько сотрудников состоят в $g,f$ и $d$ одновременно;

\item часть состоит в группах $b,d$ и $e$ одновременно;

\item и еще один человек работает в $e$ и $f$.
\end{itemize}

Собрания в разных группах можно проводить в одно и то же время, если нет сотрудников, которые в этот момент должны быть сразу на нескольких разных собраниях.

Получится ли провести все собрания в пятницу? Какое минимальное количество временных слотов необходимо? \par
\vspace{5pt}
\noindent {\bf Решение.} Задача представима в виде графа с вершинами --- группами сотрудников, а рёбра проведены между теми группами сотрудников, которые нельзя приглашать на собрание в одно и то же время. Раскраска графа в минимальное количество цветов позволяет узнать минимальное количество временных слотов.\par
\vspace{15pt}
\begin{wrapfigure}[4]{l}{0.5\linewidth} 
\vspace{-35pt}
\begin{center}
\includegraphics[scale=0.35]{img/2_company.png}
\end{center}
\end{wrapfigure}
\noindent
Слот 1 --- $a, e$ \\
Слот 2 --- $b, g$ \\
Слот 3 --- $c, f$ \\
Слот 4 --- $d$ \par
Минимальное количество цветов для раскраски равно четырём, т.\thinspaceк. граф содержит полный граф $K_4$, который можно раскрасить минимум $4$ цветами. В терминах задачи наличие $K_4$ означает, что провести собрания с сотрудниками из групп $a,b,c,d$ можно только в четыре разных временных слота.\par
\noindent {\bf Ответ.} Получится. Необходимо минимум $4$ временных слота.

\section*{Задание 3}
Напомним, что граф называется двудольным, если его можно правильно раскрасить в два цвета.\\
{\bf а)}  Какое наибольшее число ребер может быть в простом двудольном графе на $k$ белых и $m$ чёрных вершинах? (В нем не должно быть ребер, соединяющих вершины одинакового цвета.)\\
{\bf б)} Какое наибольшее количество рёбер может быть в двудольном графе на $2n$ вершинах?\par
\vspace{5pt}
\noindent {\bf Решение.} \par
\noindent{\bf а)} Наибольшее количество рёбер определяется формулой:
$$\max(|E|)=k \times m=m \times k,$$
поскольку в таком графе каждая из $k$ вершин белого цвета имеет $m$ рёбер, соединяющихся с $m$ вершинами чёрного цвета, и наоборот.\par
\vspace{5pt}
\noindent{\bf б)} Допустим, наш двудольный граф состоит из чёрных и белых вершин и число чёрных вершин $x$. Тогда белых вершин $2n-x$. Согласно  {\bf (а)}, максимальное число рёбер в таком графе представляется функцией от $x$
$$max(|E|)=f(x)=(2n-x)x=2nx-x^2.$$
Это парабола <<рогами вниз>>, которая имеет максимум в точке, являющейся решением уравнения
$$f'(x)=2n-2x=0,$$
т.\thinspaceе. $x_{max(|E|)}=n$. Подставляя это значение в $f(x)$, получаем ответ:
$$max(|E|)=(2n-n)n=n^2.$$
\noindent {\bf Ответ. (а)} --- наибольшее количество рёбер равно $m \times k$. {\bf (б)} --- наибольшее количество рёбер равно $n^2$.\par

\section*{Задание 4}
Рассмотрим алфавит, состоящий только из двух букв $a$ и $b$. Все возможные слова, которые можно получить в этом алфавите, назовем языком.\\
{\bf a)} Докажите, что в этом языке можно составить слово, в котором любая трехбуквенная комбинация этих двух букв ($aaa$, $aab$, \ldots, $bba$, $bbb$) встречается ровно один раз.\\
{\bf б)} Существует ли слово, которое удовлетворяет условию предыдущего пункта и начинается на $abba$? Если существует, то укажите его. Если не существует, то объясните, почему это невозможно.\par
\vspace{5pt}
\noindent {\bf Решение.} \par
\noindent{\bf а) Доказательство.} Данная задача представима в виде графа, в котором вершины --- уникальные трёхбуквенные комбинации. Тогда условие {\bf (а)} соответствует наличию в этом графе гамильтонова пути, в котором при переходе от каждой вершины $u$ к последующей вершине $v$ сохраняется условие: последние две буквы в трёхбуквенной комбинации $u$ совпадают с первыми двумя буквами $v$. Такой путь и соответствующее ему слово можно составить, например, таким образом:
$$(aaa,aab,abb,bbb,bba,bab,aba,baa)\rightarrow aaabbbabaa,$$
что доказывает возможность составления такого слова.\par
\noindent{\bf б)} Не существует.\par 
\noindent{\bf Объяснение.} Докажем от противного. Допустим, такое слово существует. Тогда в последующих буквах этого слова должна встретиться комбинация $bbb$. Перед этой комбинацией может быть буква $a$ или $b$:
\begin{itemize}
\item если $a$, то комбинация $abb$ встретится как минимум второй раз;
\item если $b$, то комбинация $bbb$ будет встречаться как минимум дважды.
\end{itemize}
Это противоречит условию в {\bf (а)}, поэтому такого слова не существует.
\end{document}