\documentclass[a4paper,12pt]{article}
\usepackage[utf8]{inputenc}
\usepackage{graphicx}
\usepackage{amssymb}
\usepackage{amsmath}
\usepackage[T2A]{fontenc}
\usepackage[russian]{babel}
\pagenumbering{gobble}
\usepackage{multicol}
\newcommand{\ssolve}{\par\vspace{5pt}\noindent{\bf Решение. }\par}
\newcommand{\aanswer}{\par\vspace{5pt}\noindent{\bf Ответ. }}
\newcommand{\bpm}{\begin{pmatrix}}
\newcommand{\epm}{\end{pmatrix}}
\newcommand{\bvm}{\begin{vmatrix}}
\newcommand{\evm}{\end{vmatrix}}
\newcommand{\bal}{\left\{\begin{aligned}}
\newcommand{\eal}{\end{aligned}\right.}
\newcommand{\llaq}{\quad \longrightarrow \quad}
\newcommand{\ao}[1]{\overset{#1}{\longmapsto}}

\begin{document}

\title{<<Теория вероятностей:\\ Случайная величина>>\\ \vspace{12pt} Домашнее задание №2}
\author{Байдаков Илья}
\date{\today}
\maketitle

\section*{Задание 1}
Случайная величина $X$ имеет распределение, заданное таблицей выше.
a) Дополните таблицу распределения.\\
б) Найдите таблицу распределения случайной величины $X^2$. \par
\vspace{5pt}
\begin{tabular}{|c|c|c|c|c|c|}
\hline
$X$&-2&-1&0&1&2 \\
\hline
$P$&0.05&0.1&0.15&0.2&?\\
\hline
\end{tabular}
\vspace{5pt}
\ssolve
а) Значение $P(X=2)$ найдём исходя из того, что сумма всех вероятностей при табличном задании случайной величины равняется единице:
$$P(X=2)=1-0.05-0.1-0.15-0.2=0.5,$$
то есть дополненная таблица выглядит так: \par
\vspace{5pt}
\begin{tabular}{|c|c|c|c|c|c|}
\hline
$X$&-2&-1&0&1&2 \\
\hline
$P$&0.05&0.1&0.15&0.2&0.5\\
\hline
\end{tabular} \par
\vspace{10pt}
б)Согласно заданному распределению, $X^2$ может принимать одно из трёх значений: \par
1) $X^2=0 \text{ (при $X=0$)}$ \par
2) $X^2=1 \text{ (при $X=1$ и $X=-1$)}$ \par
3) $X^2=4 \text{ (при $X=2$ и $X=-2$)}$ \par

Вероятность того, что случайная величина $X^2$ примет любое из трёх значений, равна сумме вероятностей каждого из значений $X$, при которых $X^2$ может принять это значение. Тогда, суммируя соответствующие вероятности, получаем таблицу распределения случайной величины $X^2$:\par
\vspace{5pt}
\begin{tabular}{|c|c|c|c|}
\hline
$X^2$&0&1&4 \\
\hline
$P$&0.15&0.3&0.55\\
\hline
\end{tabular}
\aanswer Таблицы приведены выше.
\section*{Задание 2}
Найдите вероятность того, что решка первый раз выпадет на нечетном по номеру
броске монетки.
\ssolve
Обозначим искомое событие (выпадение решки первый раз на на нечётном броске) как $A$. \par
Вероятность события $A$ является суммой вероятностей выпадения решки на любом $i$-м нечётном броске от $i=1$ до бесконечно большого нечётного $i$. \par
Поскольку вероятность выпадения орла и решки одинакова и равняется $P_o=P_p=1/2$, то вероятность выпадения решки на $i$-м броске равняется $P_i=(1/2)^i$. \par
С учётом этого, вероятность события $A$ находится как сумма бесконечного ряда:
$$P(A)=\left(\frac{1}{2}\right)^1+\left(\frac{1}{2}\right)^3+\left(\frac{1}{2}\right)^5+ \ldots = \sum_{n=0}^\infty \frac{1}{2^{2n+1}}=\frac{1}{2}\sum_{n=0}^\infty \frac{1}{2^{2n}}=\frac{1}{2}\sum_{n=0}^\infty \frac{1}{4^n}$$
Число под знаком суммы найдём по формуле для суммы бесконечной геометрической прогрессии:\par
$$\sum_{n=0}^\infty \frac{1}{4^n}=\frac{b_1}{1-q}, \quad b_1 = \frac{1}{4^0}=1 ,\quad  q=\frac{1}{4} \llaq \sum_{n=0}^\infty \frac{1}{4^n}=\frac{4}{3}$$ 
Таким образом,
$$P(A)=\frac{1}{2}\sum_{n=0}^\infty \frac{1}{4^n} = \frac{1}{2} \cdot \frac{4}{3} = \frac{2}{3}$$
\aanswer $P = \frac{2}{3}$.
\section*{Задание 3}
Случайная величина $X$ имеет функцию распределения, указанную на графике.
Найдите вероятности событий.
\ssolve 
а) $P(X=1) = \text{величина скачка} = \frac{1}{4} - \frac{1}{8} = \frac{1}{8}$ \par
б) $P(X=2) = \text{значение в непрерывной области } F(x) = 0$ \par
в) $P(X \in \left( 1 \frac{1}{2}; 2 \right]) = F(2) - F(1 \frac{1}{2}) = \frac{1}{2} - \frac{1}{2} = 0$ \par
г) $P(X \in \left( 1 ; 2 \right]) = F(2) - F(1) = \frac{1}{2} - \frac{1}{4} = \frac{1}{4}$ \par
д) $P(X \in \left[ 1 ; 2 \right]) = F(2) - F(1) + F(1) = \frac{1}{2} - \frac{1}{4} + \frac{1}{8} = \frac{3}{8}$ \par
\section*{Задание 4}
Пусть количество ссылок на случайно выбранном сайте имеет распределение Парето с параметрами $x_m = 1$; $k = 1.1$. \par
а) Какую плотность имеет случайная величина, равная количеству ссылок на случайно выбранный сайт? \par
б) Найдите вероятность того, что на сайте будет не более пяти ссылок.
\ssolve
а) Случайная величина, равная количеству ссылок на случайно выбранный сайт, имеет плотность распределения Парето:
$$f_X(x)=
\left\{
\begin{aligned}
& \frac{k x_m^k}{x^{k+1}}, \quad x \geq x_m\\
& 0 , \qquad x < x_m \\
\end{aligned}
\right.
$$
б) Используя определение функции распределения Парето, \par
$$F_X(x) = 1 - \left( \frac{x_m}{x} \right)^k , \quad P(X \leq 5) = F(5) = 1 - \frac{1}{5}^{1.1} \simeq 0.83 $$
\aanswer а) плотность распределения Парето, б) $P \simeq 0.83$.
\section*{Задание 5}

График функции плотности случайной величины X изображен на рисунке.
Какова величина $P(X \in [ \frac{1}{2};  2 ] )$? \par
\includegraphics[width=0.5 \textwidth]{gr_5.png}
\ssolve
Согласно известной формуле,
$$P(X \in (a ; b ] ) = \int_a^b f_X(x) dx = \text{площадь по графиком $f_X(x)$ на  $(a, b]$}.$$
В нашем случае, считая площадь одной клетки равной $\frac{1}{8}$,
$$P(X \in [a ; b ] ) = \frac{1}{8} \cdot 3 + \frac{1}{8} \cdot \frac{3}{2} = \frac{9}{16}.$$
\aanswer $P = \frac{9}{16}$.
\section*{Задание 6}
Считается, что длительность телефонного разговора подчиняется показательному закону. Пусть установлено, что разговор продлится более $5$ минут с вероятностью $25$. \par
а) Чему равняется параметр $\lambda$? \par
б) В условиях предыдущей задачи найдите вероятность того, что разговор продлится не дольше $10$ минут.
\ssolve
Функция показательного распределения:
$$F_X(x)=
\left\{
\begin{aligned}
& 1 - e^{-\lambda x}, \quad x \geq 0\\
& 0 , \qquad \qquad x < 0 \\
\end{aligned}
\right.$$
По условию, 
$$P(X > 5) = \frac{2}{5} \llaq P(X \leq 5) = 1 - P(X > 5) = 1 - \frac{2}{5} = \frac{3}{5} $$
Далее, используя $F_X(x)$,
$$P(X \leq 5) = F_X(5) = 1 - e^{-5 \lambda} = \frac{3}{5} \llaq$$
$$e^{-5\lambda} = \frac{2}{5} \llaq -5\lambda = \ln \frac{2}{5} \llaq \lambda = -\ln \frac{2}{5} \cdot \frac{1}{5} \simeq 0.18.$$
б) По определению,
$$P(X \leq 10) = F(10) = 1-e^{-\lambda \cdot 10} \simeq 0.84.$$
\aanswer 
а) $\lambda \simeq 0.18$ \qquad
б) $P(X \leq 10) \simeq 0.84.$ \par
\section*{Задание 7}
Случайная величина $X$ имеет стандартное равномерное распределение (т. е. $X \sim U[0; 1])$. \par
а) Какое распределение будет иметь случайная величина \\ $Y = (X + 1) \cdot 2$? \par
б) Найдите функцию распределения и функцию плотности случайной величины $Z = \ln(X + 1)$.
\ssolve
а) По определению,
$$F_Y(x) = P(Y \leq x) = P((X+1)\cdot 2 \leq x) = P(X \leq \frac{x-2}{4-2}) \llaq $$
$$\llaq F_Y(x) \sim U[2,4]$$
Значит, случайная величина $Y$ имеет равномерное распределение на отрезке $[2, 4]$. \par
\vspace{5pt}
б) По определению,
$$F_Z(x) = P(\ln(X+1) \leq x ) = P(X \leq e^x - 1)$$
Из условия задачи $X \in [0;1]$, найдём границы $x$ для распределения $F_Z(x)$:
$$e^x - 1 \geq 0 \llaq x \geq 0,$$
$$ e^x - 1 \leq 1 \llaq x \leq \ln 2$$
Получаем распределение случайной величины $Z$:
$$F_Z(x)=
\left\{
\begin{aligned}
& e^x - 1, \quad x \in [0; \ln 2] \\
& 0 , \qquad \qquad x < 0 \\
& 1 , \qquad \qquad x > \ln 2 \\
\end{aligned}
\right.$$
Функция плотности случайной величины $Z$ получается дифференцированием $F_Z(x)$ на том же интервале:
$F'_Z(x) = e^x$. \par
Значит, имеем $\rho_Z(x)$:
$$\rho_Z(x)=
\left\{
\begin{aligned}
& e^x , \quad x \in [0; \ln 2] \\
& 0 , \qquad x \notin [0; \ln 2] \\
\end{aligned}
\right.$$
\aanswer
а) $Y \sim U[2;4] \qquad$ б) $F_Z(x)$ и $\rho_Z(x)$ приведены выше.
\end{document}