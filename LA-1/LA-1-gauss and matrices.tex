\documentclass[a4paper,12pt]{article}
\usepackage[utf8]{inputenc}
\usepackage{graphicx}
\usepackage{amssymb}
\usepackage{amsmath}
\usepackage[T2A]{fontenc}
\usepackage[russian]{babel}
\pagenumbering{gobble}
\usepackage{wrapfig}


\begin{document}

\title{<<Линейная алгебра:\\ метод Гаусса и матрицы>>\\ \vspace{12pt} Домашнее задание}
\author{Байдаков Илья}
\date{\today}
\maketitle


\section*{Задание 1}
Решите систему линейных уравнений:
$$
\left\{
\begin{aligned}
&6x + 12y + 5z + t = -6\\
&9x + 18y + 17z -8t = -9\\
&5x + 10y + 4z + t = -5
\end{aligned}
\right.
$$

\noindent{\bf Решение.} Запишем расширенную матрицу системы:

$$ \left(\begin{array}{cccc|c}
6 & 12 & 5 & 1 & -6 \\
9 & 18 & 17 & -8 & -9 \\
5 & 10 & 4 & 1 & -5
\end{array} \right)
\overset{I-III}{\longmapsto}
\left(\begin{array}{cccc|c}
1 & 2 & 1 & 0 & -1 \\
9 & 18 & 17 & -8 & -9 \\
5 & 10 & 4 & 1 & -5
\end{array} \right)
\longmapsto \\ $$

$$ \overset{III-5I}{\overset{II-9\cdot I}{\longmapsto}}
\left(\begin{array}{cccc|c}
1 & 2 & 1 & 0 & -1 \\
0 & 0 & 8 & -8 & 0 \\
0 & 0 & -1 & 1 & 0
\end{array} \right)
\overset{II+8\cdot III}{\longmapsto}
\left(\begin{array}{cccc|c}
1 & 2 & 1 & 0 & -1 \\
0 & 0 & 0 & 0 & 0 \\
0 & 0 & -1 & 1 & 0
\end{array} \right).\\ $$
Имеем $x$ и $z$ главными неизвестными, а $y$ и $t$ --- свободными. Полученная матрица соответствует системе уравнений:
$$
\left\{
\begin{aligned}
&x + 2y + z = -1\\
&z = t
\end{aligned}
\right.
\qquad \Leftrightarrow \qquad
\left\{
\begin{aligned}
&x = -2y - t -1\\
&z = t
\end{aligned}
\right. .
$$
\vspace{1ex}
\noindent {\bf Ответ.} $x = -2y - t -1$, $z = t$, где $y, t \in \mathbb{R}$.

\section*{Задание 2}
Система линейных уравнений называется {\it совместной}, если она имеет хотя бы одно решение (в вещественных числах).
Определите, при каких значениях параметра $\lambda \in \mathbb{R}$ следующая система является совместной, а при каких --- нет
({\it явное решение системы предъявлять не обязательно}):
$$\left\{
\begin{aligned}
&8x_1 + 6 x_2 + 3 x_3 +2x_4 = 5\\
-&12x_1 - 3x_2 -3x_3 + 3x_4 = -6\\
&4x_1 +5x_2 +2x_3 + 3x_4 = 3\\
& \lambda x_1 + 4x_2 + x_3 + 4x_4= 2
\end{aligned}
\right.$$
\noindent{\bf Решение.} Запишем расширенную матрицу системы:
$$ \left(\begin{array}{cccc|c}
8 & 6 & 3 & 2 & 5 \\
-12 & -3 & -3 & 3 & -6 \\
4 & 5 & 2 & 3 & 3 \\
\lambda & 4 & 1 & 4 & 2 \\
\end{array} \right) 
\overset{I-2\cdot III}{\longmapsto}
\left(\begin{array}{cccc|c}
0 & -4 & -1 & -4 & -1 \\
-12 & -3 & -3 & 3 & -6 \\
4 & 5 & 2 & 3 & 3 \\
\lambda & 4 & 1 & 4 & 2 \\
\end{array} \right) 
\longmapsto $$

$$ 
\overset{I+IV}{\longmapsto}
\left(\begin{array}{cccc|c}
\lambda & 0 & 0 & 0 & 1 \\
0 & 12 & 3 & 12 & 15 \\
4 & 5 & 2 & 3 & 3 \\
\lambda & 4 & 1 & 4 & 2 \\
\end{array} \right).$$
Таким образом, из полученной матрицы следует, что 
$$\lambda = 1.$$
Дальнейшими преобразованиями с учётом $\lambda = 1$ система может быть приведена к ступенчатому виду. Значит $\lambda = 1$ является условием совместности системы. \par
\vspace{1ex}
\noindent{\bf Ответ.} Система совместна при $\lambda = 1$ и несовместна при $\lambda \neq 1$.

\section*{Задание 3}
Найдите многочлен $f(x)$ третьей степени, для которого 
\[
f(1) = 1, \ f(-1) = 13, \ f(2) = 7, \ f(-3) = 17.
\]
\noindent {\bf Решение.} 
Общий вид искомого многочлена $f(x)$:
$$ax^3 + bx^2 + cx + d = m.$$
Подставляя $x$ в этот многочлен, составим систему из четырёх линейных уравнений, в которых $a$, $b$, $c$, $d$ --- неизвестные. Запишем и преобразуем расширенную матрицу такой системы: 
$$
\left(\begin{array}{cccc|c}
1 & 1 & 1 & 1 & 1 \\
-1 & 1 & -1 & 1 & 13 \\
8 & 4 & 2 & 1 & 7 \\
-27 & 9 & -3 & 1 & 17 \\
\end{array} \right) 
\overset{II+I}{\overset{III-8\cdot I}{\longmapsto}}
\left(\begin{array}{cccc|c}
1 & 1 & 1 & 1 & 1 \\
0 & 2 & 0 & 2 & 14 \\
0 & -4 & -6 & -7 & -1 \\
-27 & 9 & -3 & 1 & 17 \\
\end{array} \right) 
\longmapsto $$

$$ \overset{I-II\cdot \frac{1}{2}}{\overset{III+2\cdot II}{\longmapsto}}
\left(\begin{array}{cccc|c}
1 & 1 & 1 & 1 & 1 \\
-1 & 1 & -1 & 1 & 13 \\
0 & 0 & -6 & -3 & 27 \\
-27 & 9 & -3 & 1 & 17 \\
\end{array} \right)
\overset{VI+27 \cdot I}{\overset{III \cdot \frac{1}{3}}{\longmapsto}}
\left(\begin{array}{cccc|c}
1 & 0 & 1 & 0 & -6 \\
0 & 2 & 0 & 2 & 14 \\
0 & 0 & -2 & -1 & 9 \\
0 & 9 & 24 & 1 & -145 \\
\end{array} \right) 
\longmapsto 
$$


$$ \overset{IV-\frac{II \cdot \frac{9}{2}}{8}}{\longmapsto}
\left(\begin{array}{cccc|c}
1 & 1 & 1 & 1 & 1 \\
0 & 2 & 0 & 2 & 14 \\
0 & 0 & -2 & -1 & 9 \\
0 & 0 & 24 & -8 & -208 \\
\end{array} \right)
\overset{\frac{VI+12 \cdot III}{-20}}{\longmapsto}
\left(\begin{array}{cccc|c}
1 & 0 & 1 & 0 & -6 \\
0 & 2 & 0 & 2 & 14 \\
0 & 0 & -2 & -1 & 9 \\
0 & 0 & 0 & 1 & 5 \\
\end{array} \right) 
\longmapsto 
$$

$$ \overset{II \cdot \frac{1}{2}}{\longmapsto}
\left(\begin{array}{cccc|c}
1 & 0 & 1 & 0 & -6 \\
0 & 1 & 0 & 1 & 7 \\
0 & 0 & -2 & -1 & 9 \\
0 & 0 & 0 & 1 & 5 \\
\end{array} \right)
\overset{\frac{III+IV}{-\frac{1}{2}}}{\overset{II-IV}{\longmapsto}}
\left(\begin{array}{cccc|c}
1 & 0 & 1 & 0 & -6 \\
0 & 1 & 0 & 0 & 2 \\
0 & 0 & 1 & 0 & -7 \\
0 & 0 & 0 & 1 & 5 \\
\end{array} \right)
\longmapsto
$$
$$ \overset{I -III}{\longmapsto}
\left(\begin{array}{cccc|c}
1 & 0 & 0 & 0 & 1 \\
0 & 1 & 0 & 0 & 2 \\
0 & 0 & 1 & 0 & -7 \\
0 & 0 & 0 & 1 & 5 \\
\end{array} \right) .
$$
Из преобразованной матрицы следует решение системы уравнений:
$$
\left\{
\begin{aligned}
&a = 1\\
&b = 2\\
&c = -7 \\
&d = 5 .\\
\end{aligned}
\right.
$$

Значит, искомый многочлен имеет вид
$$f(x) = x^3 + 2x^2 -7x + 5.$$
\vspace{1ex}
\noindent{\bf Ответ.} $f(x) = x^3 + 2x^2 -7x + 5$.
\newpage
\section*{Задание 4}
Вычислите:
\[
(A^TB^T + B^TA)^T - ((2BA)^T - E^3)^2 - A^T B - BA,
\]
где 
\[
A = \begin{pmatrix}
1 & 2\\
-3 & 0\\
\end{pmatrix},\
B = \begin{pmatrix}
0 & -1\\
2 & 1\\
\end{pmatrix}.
\]
\noindent {\bf Решение.}\\ Преобразуем выражение:
\begin{align*}
&(A^TB^T + B^TA)^T - ((2BA)^T - E^3)^2 - A^T B - BA \Leftrightarrow \\
&((BA)^T+B^TA)^T-((2BA)^T-E^3)^2-A^TB-BA \Leftrightarrow \\
&((BA)^T)^T+(B^TA)^T-((2BA)^T-E^3)^2-A^TB-BA \Leftrightarrow \\
&BA+A^T(B^T)^T-((2BA)^T-E^3)^2-A^TB-BA \Leftrightarrow \\
&BA+A^TB-((2BA)^T-E^3)^2-A^TB-BA \Leftrightarrow \\
&-(2(BA)^T-E^3)^2
\end{align*}
Теперь вычислим:
$$ BA = 
\begin{pmatrix}
0 & -1\\
2 & 1\\
\end{pmatrix}
\begin{pmatrix}
1 & 2\\
-3 & 0\\
\end{pmatrix}
=
\begin{pmatrix}
3 & 0\\
-1 & 4\\
\end{pmatrix}. $$
Далее,
$$ (BA)^T=
\begin{pmatrix}
3 & -1\\
0 & 4\\
\end{pmatrix}, \\
 2(BA)^T =
\begin{pmatrix}
6 & -2\\
0 & 8\\
\end{pmatrix}$$
С лекции известно, что:
$$E^3 = 
\begin{pmatrix}
1 & 0\\
0 & 1\\
\end{pmatrix}.$$
$$(2BA)^T-E^3 =
\begin{pmatrix}
5 & -2\\
0 & 7 \\
\end{pmatrix}, $$
$$((2BA)^T-E^3)^2 = 
\begin{pmatrix}
5 & -2\\
0 & 7 \\
\end{pmatrix}
\begin{pmatrix}
5 & -2\\
0 & 7 \\
\end{pmatrix} =
\begin{pmatrix}
25 & -24\\
0 & 49 \\
\end{pmatrix}.$$
Окончательно:
$$-((2BA)^T-E^3)^2=
\begin{pmatrix}
-25 & 24\\
0 & -49 \\
\end{pmatrix}.$$
\noindent{\bf Ответ.} $\begin{pmatrix}
-25 & 24\\
0 & -49 \\
\end{pmatrix}.$
\newpage
\section*{Задание 5}
Найдите все матрицы, коммутирующие с матрицей
$\begin{pmatrix}
-1 & 3\\
2 & 5
\end{pmatrix}.$
\par
\noindent {\bf Решение.}\\ Запишем элементы искомой матрицы в виде переменных. Она должна быть той же размерности, что и данная. Тогда верно, что матрица:
$$\begin{pmatrix}
-1 & 3\\
2 & 5 \\
\end{pmatrix}
\begin{pmatrix}
a & b\\
c & d \\
\end{pmatrix}
 =
\begin{pmatrix}
-a + 3c & -b+3d \\
2a + 5c & 2b+5d \\
\end{pmatrix}
$$
совпадает с матрицей
$$ \begin{pmatrix}
a & b\\
c & d \\
\end{pmatrix}
\begin{pmatrix}
-1 & 3\\
2 & 5 \\
\end{pmatrix}
 =
\begin{pmatrix}
-a + 2b & 3a+5b \\
-c + 2d & 3c+5d \\
\end{pmatrix},
$$
т.е.
$$
\begin{pmatrix}
-a + 3c & -b+3d \\
2a + 5c & 2b+5d \\
\end{pmatrix} = 
\begin{pmatrix}
-a + 2b & 3a+5b \\
-c + 2d & 3c+5d
\end{pmatrix}.$$
Приравняем соответствующие элементы матриц и получим систему уравнений: \par
$$\left\{
\begin{aligned}
&-a+3c = -a +2b\\
&-b+3d=6a+5b\\
&2a+5c=-c+2d\\
&2b+5d=3c+5d
\end{aligned}
\right.
\qquad
\Leftrightarrow
\qquad 
\left\{
\begin{aligned}
& 3c = 2b\\
&3d=3a+6b\\
\end{aligned}
\right.
$$
Принимая, $a$ и $b$ главными неизвестными, а $c$ и $d$ --- свободными, получаем:
$$\left\{
\begin{aligned}
&c = \frac{2}{3}b\\
&d = a+2b\\
\end{aligned}
\right.
$$
Искомые матрицы:
$$ \begin{pmatrix}
a & b\\
\frac{2}{3}b & a+2b \\
\end{pmatrix},$$
где $a$, $b \in \mathbb{R}.$ \par
\noindent{\bf Ответ.}
$ \begin{pmatrix}
a & b\\
\frac{2}{3}b & a+2b \\
\end{pmatrix},$
$a$, $b \in \mathbb{R}.$
\newpage
\section*{Задание 6}
Для произвольных $n \times n$ матриц $A$ и $B$ предложите способ вычисления выражения
$$3BA^4 + B^2A^3 - 3BA^3B - B^2A^2B,$$
в котором используется лишь константное (не зависящее от $n$) количество указанных выше операций, и при этом {\bf не более~4-х} операций умножения матрицы на матрицу. Разрешается выделять дополнительную память --- хранить в памяти константное число $n \times n$ матриц (возможно, зависящих от $A, B$), которые могут быть использованы при промежуточных вычислениях.\par
\noindent {\bf Решение.}\\
Преобразуем выражение:
\begin{align*}
&3BA^4 + B^2A^3 - 3BA^3B - B^2A^2B \Leftrightarrow \\
&3BA^3(A-B) + B^2A^2(A-B) \Leftrightarrow \\
&3BA \cdot A^2(A-B) + B^2 \cdot A^2(A-B) \Leftrightarrow \\
&(3BA+B^2)A^2(A-B) \Leftrightarrow \\
&B(3A+B)A^2(A-B).
\end{align*}
Тогда вычисление предлагается провести так: 
\begin{itemize}
\item Выделить память для хранения $A$, $B$, $3A$, $(3A+B)$, $A^2$ и $(A-B)$
\item Вычислить $3A$, $(3A+B)$ и $(A-B)$
\item Вычислить $A^2$ (1-я операция умножения)
\item Вычислить $B\cdot (3A+B)\cdot A^2 \cdot (A-B)$ (2-я, 3-я и 4-я операции умножения)
\end{itemize} 
\newpage
\section*{Задание 7}
Студент перемножил следующие матрицы, расположив их в некотором порядке (были использованы все матрицы):
$$
A = 
\begin{pmatrix}
5 & 1 & 2 & -1 & 0\\
1 & 4 & -3 & 2 & 2
\end{pmatrix}, \
B =\begin{pmatrix}
1 & 1 & 1\\
-1 & 0 & 3\\
-2 & 1 & 4\\
0 & 0 & 3
\end{pmatrix}
$$
$$
C = \begin{pmatrix}
1 & 2 & -1 & 0\\
-3 & 0 & 1 & 2
\end{pmatrix},\
D = 
\begin{pmatrix}
4 & -1\\
1 & 0
\end{pmatrix}, \
F = \begin{pmatrix}
-4 & 1\\
12 & -7\\
1 & 1
\end{pmatrix}.
$$
Вычислите матрицу, которую он получил (перечислите все возможные варианты).\par
\noindent {\bf Решение.}\\
Составим орграф, в котором вершины --- размер $m \times n$ матрицы, а рёбра соединяют те вершины, для которых определено умножение матриц.\par
При этом заметим, что матрица с уникальным количеством столбцов или строк только одна, это $A$. Значит, в произведении нескольких матриц она должна быть последней, т.к. в противном случае ей не найдётся сомножителя. Поэтому рёбра начнём строить от $A$ в обратном порядке. Граф показан ниже. Рёбра подписаны размером матрицы, получившейся при перемножении на предыдущем шаге. \par
\centerline{\includegraphics[scale=0.7]{img/graph.png}}
Таким образом, есть две последовательности, в которые можно поставить матрицы для перемножения. Порядок перемножения не рассматривается, т.к. по условию для ответа требуется только вычисленная матрица. Возможные варианты последовательностей и вычисленных матриц:
\begin{itemize}
\item $D \cdot C \cdot B \cdot F \cdot A=
\begin{pmatrix}
-35 & -7 & -14 & 7 & 0\\
-1 & 15 & -14 & 9 & 8
\end{pmatrix}$
\item $C \cdot B \cdot F \cdot D \cdot A=
\begin{pmatrix}
1 & 4 & -3 & 2 & 2\\
137 & 16 & 65 & -34 & -6
\end{pmatrix}$
\end{itemize}
\end{document}