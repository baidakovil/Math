\documentclass[a4paper,12pt]{article}
\usepackage[utf8]{inputenc}
\usepackage{graphicx}
\usepackage{amssymb}
\usepackage{amsmath}
\usepackage[T2A]{fontenc}
\usepackage[russian]{babel}
\pagenumbering{gobble}


\begin{document}

\title{<<Дискретная математика: \\ неориентированные графы>> \\ \vspace{12pt} Домашнее задание}
\author{Байдаков Илья}
\date{\today}
\maketitle

\section*{Задание 1}
\begin{center}
\includegraphics[scale=1]{img/graphHW.pdf}
\end{center}
\vspace{-15 pt}
Граф $G$ изображен на рисунке выше.\par
{\bf а)} Найдите максимальную длину простого цикла в графе $G$. Укажите
все различные простые циклы максимальной длины.\par
{\bf б)} Верно ли, что если в графе $G$ удалить любое ребро, то из любой его
вершины можно будет добраться до любой? При положительном ответе
приведите обоснование, при отрицательном --- укажите ребро, которое
можно удалить, и вершины, между которыми не будет пути.\par
{\bf в)} Какое минимальное количество рёбер необходимо удалить из графа $G$, чтобы он стал несвязным?\par
\vspace{15 pt}
\noindent {\bf Решение.}
\begin{itemize}
\item[{\bf а)}] Максимальная длина простого цикла в графе $G$ равна $4$. Это циклы $ABCDA$ и $CEFHC$.\par
\item[{\bf б)}] Верно, если в графе $G$ удалить любое ребро, то из любой его вершины можно будет добраться до любой вершины. \par
{\bf Обоснование.} Приведём три утверждения.\par
{\bf Утв. 1.} Каждое ребро графа $G$ принадлежит простому циклу (доказывается перебором всех рёбер). \par
{\bf Утв. 2.} Условие <<Из любой вершины графа можно будет добраться до любой вершины>> соответствует условию <<Граф будет связным>> по определению связности. \par
{\bf Утв. 3.} <<Если в графе есть простой цикл, удалим из него произвольное ребро. Граф останется связным>> (известно с лекции).\par
Согласно утверждению 3, с учётом утв. 1 и 2, из любой вершины можно будет добраться до любой вершины при удалении любого ребра. \par
\item[{\bf в)}] Минимальное количество рёбер, которое необходимо удалить, чтобы граф $G$ стал несвязным: $2$. Это следует из пункта {\bf (б)} и из примера удаления двух рёбер, например, $BC$ и $DC$, при которых граф становится несвязным. 
\end{itemize}

\section*{Задание 2}
В государстве $100$ городов, и из каждого из них выходит $4$ дороги в другие города этого государства. Сколько всего дорог в государстве?\par
\vspace{15 pt}
{\bf Решение.} Условие задачи можно представить в виде графа $G$, где дороги --- рёбра, а города --- вершины. Согласно лемме о рукопожатиях,
$$\sum_{v_i\in V}\deg v_i=2|E|.$$
Таким образом, число дорог
$$|E|=\frac{\displaystyle \sum_{v_i\in V}\deg v_i}{2}=\frac{4\cdot100}{2}=200$$
{\bf Ответ.} 200 дорог.\par
\newpage

\section*{Задание 3}
Можно ли нарисовать картинки на рисунке ниже, не отрывая карандаша от бумаги и проходя по каждой линии по одному разу?

Если можно, то покажите, как это сделать.

Если нельзя, то докажите, что это сделать невозможно.
\vspace{-40 pt}
\begin{center}
\includegraphics[scale=1]{img/graphHW2.pdf}
\end{center}
{\bf Решение.}
\begin{itemize}
\item[{\bf а)}] Можно. Картинка представима в виде графа, и тогда условие <<Не отрывая карандаша от бумаги и проходя по каждой линии по одному разу>> соответствует <<В графе существует эйлеров путь или эйлеров цикл>>.\par
По критерию существования эйлерова цикла, эйлеров цикл существует тогда и только тогда, когда все вершины имеют четную степень. В данном графе все вершины имеют чётную степень. \par Как сделать: провести эйлеров цикл $ABADGHGFHEFDCEBCA$, где $AB$, $GH$ --- <<внешние>> рёбра графа, а $BA$, $HG$ --- <<внутренние>>.
\item[{\bf б)}] Нельзя. Доказательство. Картинка представима в виде графа, у которого количество вершин с нечётной степенью больше двух, а это не удовлетворяет критериям существования эйлерова цикла и эйлерова пути.
\end{itemize}
\newpage
\section*{Задание 4}
В дереве на $2021$ вершинах ровно три вершины имеют степень $1$.
Сколько вершин имеют степень $3$? \par
{\bf Решение.} Дерево содержит только одну вершину степени $3$. Докажем от обратного.\par Допустим, вершин со степенью $\deg v = 3$ две или больше двух. Поскольку мы имеем дерево, то каждое из двух рёбер этих вершин является основанием ветвей, а у каждой ветви есть как минимум одна висячая вершина. Тогда минимальное суммарное число висячих вершин в дереве равно произведению <<2 $\times$ число вершин со степенью 3>>, т.е. больше или равно четырём, что противоречит условию задачи. Значит, исходное предположение было неверным. \par
{\bf Ответ.} Одна вершина имеет степень $3$.

\section*{Задание 5}
У некоторого графа на $6$ вершинах ровно $11$ ребер. Докажите, что этот граф связен. \par
{\bf Доказательство.} Предположим, что граф не связен. Рассмотрим возможные случаи:
\begin{itemize}
\item Граф имеет две компоненты связности, c 5 и 1 вершинами. Тогда по формуле для числа рёбер в полном графе для $K_5$ имеем:
$$K_5=\frac{n(n-1)}{2}=10,$$
т.е. первая компонента связности имеет не более $10$ рёбер, а вторая не более $0$ рёбер, т.е. условие задачи не может быть выполнено.
\item Граф имеет две компоненты связности, с 4 и 2 вершинами. Будет не более $7$ рёбер, т.е. условие не выполнено.
\item Граф имеет две компоненты связности по 3 вершины. Будет не более $6$ рёбер, т.е. условие не выполнено.
\item Граф имеет три компоненты связности по 2 вершины. Будет не более $3$ рёбер, т.е. условие не выполнено.
\item Граф имеет шесть компонент связности по 1 вершине в каждой. Тогда рёбер нет.
\end{itemize}
Мы пришли к противоречию. Значит, исходное предположение не верно и данный граф связен.

\section*{Задание 6}
Можно ли за несколько ходов (по шахматным правилам и не выходя за пределы доски $3\times 3$) поставить коней так, чтобы из расположения на левой картинке получилось расположение коней на правой?

Если можно, то укажите последовательность шагов.

Если нельзя, то докажите, что это сделать невозможно.

\begin{center}
\includegraphics[scale=1.5]{img/Knights1.pdf}\qquad \qquad \includegraphics[scale=1.5]{img/Knights2.pdf}
\end{center}

Поставить коней так, как указано в условии задачи, нельзя. \par
{\bf Доказательство.} Задача представима в виде графа, где каждой клетке, на которую может перейти любой из коней, соответствуют вершины, а рёбрам вершины соответствуют возможные ходы коня из этой клетки. Для исходного и требуемого расположения соответствуют такие графы (по центру и справа):
\begin{center}
\includegraphics[scale=0.7]{img/Knights_solve.png}
\end{center}
Поскольку кони могут двигаться по графу только на смежные вершины, то для перехода из исходного положения в требуемое как минимум одному из коней нужно <<перескочить>> соседнего коня или занять его место. Это не соответствует шахматным правилам. Других путей нет, значит, такой переход неосуществим.


\end{document}