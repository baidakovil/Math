\documentclass[a4paper,12pt]{article}
\usepackage[utf8]{inputenc}
\usepackage{graphicx}
\usepackage{amssymb}
\usepackage[russian]{babel}
\pagenumbering{gobble}

\begin{document}

\title{<<Дискретная математика: \\ множества и логика>> \\ \vspace{12pt} Домашнее задание}
\author{Байдаков Илья}
\date{\today}
\maketitle


\section*{Задание 1}
Какие из следующих равенств выполнены для любых множеств $A$, $B$ и $C$?
Если равенство верно, то докажите его. Если не выполнено, то приведите контрпример.
\begin{itemize}
\item[a)]
$A\setminus (A\cap B)= A \cap(A \setminus B)$.\\
Равенство верно, выполняется для любых множеств $A$ и $B$. \\
{\bf Доказательство.} Заменим множества $A$ и $B$ на высказывания $a$, $b$; операции со множествами на операции с высказываниями:
\begin{itemize}
 \item $A\cap B$ заменим на $a \wedge b$,
 \item $A\setminus B$ заменим на $\neg (a \rightarrow b)$ (известно с занятия), и так далее. 
 \end{itemize}
Выражение приводится к виду: \\
$$a \wedge (\neg (a \wedge b)) = a \wedge (\neg (a \rightarrow b))$$
По виду выражения можно утверждать, что проверить его верность можно, проверив верность выражения \\
$$a \wedge x = a \wedge y	, \qquad (*)$$
где $x = \neg (a \wedge b)$, a $y = \neg (a \rightarrow b)$. \\

Для этого воспользуемся таблицей истинности: \\
\nopagebreak

\begin{tabular}{|c|c|c|c|c|c|}
\hline
$a$ &$b$ & $x = \neg (a \wedge b)$ & $y = \neg (a \rightarrow b)$ & $a\wedge x$ &  $a\wedge y$ \\
\hline
0&0&1&0&0&0 \\
\hline
0&1&1&0&0&0 \\
\hline
1&0&1&1&1&1 \\
\hline
1&1&0&0&0&0 \\
\hline
\end{tabular}\\

Выражение (*) верно для всех пар значений $a, b$, следовательно, исходное выражение тоже верно для любых множеств $A$ и $B$.

\item[б)] 
 $(A\cup B)\triangle (A\cap B)= A \triangle B;$\\
Равенство верно, выполняется для любых множеств $A$ и $B$. \\ {\bf Доказательство.} Построим диаграмму Эйлера-Венна для левой части равенства:\\
\begin{figure}[h]
\centering
\includegraphics[width=1\textwidth]{eyler_b.png}
\end{figure}

Из диаграммы видно, что область значений  $(A\cup B)\triangle (A\cap B)$ совпадает с областью значений $A \triangle B$ (последнее не обозначено на диаграмме отдельно). Это значит, что равенство верно.
\item[в)] 
 $((A \backslash B) \cup(A \backslash C)) \cap(A \backslash(B \cap C))=A \backslash(B \cup C).$ \\
 Равенство не верно. \\ {\bf Доказательство.} Для контрпримера возьмём множества \\
 $A = \{1,2\}, B = \{2,3\}, C = \{3,4\}.$ \\
 Тогда
$$ (\{1\} \cup\{1,2\}) \cap (\{1,2\} \backslash \{3\})=\{1,2\} \backslash \{2,3,4\} ; $$
$$ \{1,2\} \cap \{1,2\} \neq \{1\}, $$
значит, равенство не выполнено для любых множеств $A$, $B$ и $C$.\\
\end{itemize}
{\bf Ответ:} Равенства (а), (б) выполнены для любых множеств $A$, $B$ и $C$.\\
\section*{Задание 2}
Верно ли, что для любых множеств $A$ и $B$ выполняется включение

$$(A\cup B)\setminus B \subseteq A?$$
Верно, включение выполняется для любых $A$ и $B$. \\
{\bf Решение:} \\
по определению, $A\cup B$ помимо всех элементов $A$ содержит все элементы $B$. Так же по определению, $(A\cup B)\setminus B$ содержит все элементы из $A\cup B$, кроме элементов, содержащихся в $B$. \\
Возможны характерные случаи: 
\begin{itemize}
\item Eсли $A$ и $B$ имеют общие элементы, то $(A\cup B)\setminus B$ будет иметь меньше элементов, чем $A$, но принадлежащих $A$;
\item Eсли $A$ и $B$ состоят из одинаковых элементов, и/или \\
 $A$ — пустое множество, \\
 то $(A\cup B)\setminus B$ будет пустым множеством;
\item Eсли $A$ и $B$ не имеют общих элементов, $(A\cup B)\setminus B$ будет состоять только из элементов $A$.
\end{itemize}
Таким образом, в каждом из возможных случаев включение будет выполняться.

\section*{Задание 3}
Докажите, что $\neg (a\vee (b \oplus 1))\wedge (a \rightarrow 1)=\neg a \wedge b.$ \\
{\bf Доказательство.} \\
Подстановкой констант в $b \oplus 1$ получаем $b \oplus 1 = \neg b$. \\
Согласно таблице истинности, $a \rightarrow 1 = 1$. \\
Получаем:
$$\neg (a\vee \neg b)\wedge 1 =\neg a \wedge b.$$
По закону поглащения, $x \wedge 1 = x$:
$$\neg (a\vee \neg b) =\neg a \wedge b.$$
По законам Моргана, $\neg(a \vee b) = \neg a \wedge \neg b$; $\neg (\neg b) = b$:
$$\neg a \wedge b = \neg a \wedge b.$$

\section*{Задание 4}
Для каких из ниже приведенных чисел ложно высказывание: <<Число четно $\wedge$ (В числе $7$ цифр $ \rightarrow \neg$(Третий разряд числа четный))>>? \\

{\bf а)} $0$ \qquad  {\bf б)} $1234567,$ \qquad {\bf в)} $2222222,$  \qquad {\bf г)} $123457.$ \\

{\bf Решение:} Упростим высказывание. Для этого выведем вспомогательную формулу $$b \rightarrow \neg c=\neg (b \wedge c). \quad (*)$$
Формула (*) следует из таблицы истинности:
\\

\begin{tabular}{|c|c|c|c|}
\hline$a$ & $b$ & $a\rightarrow \neg b$ & $\neg (a \wedge b)$ \\
\hline 0 & 0 & 1 & 1 \\
\hline 0 & 1 & 1 & 1 \\
\hline 1 & 0 & 1 & 1 \\
\hline 1 & 1 & 0 & 0 \\
\hline
\end{tabular}
\\

Обозначим высказывания:

a$=$<<Число чётно>>, 

b$=$<<В числе $7$ цифр>>,

c$=$<<Третий разряд числа чётный>>.

Получаем формулу $a\wedge (b \rightarrow \neg c$). Из (*) следует:

$$a\wedge (b \rightarrow \neg c)=a\wedge \neg( b \wedge c).$$

Нам необходимо указать те числа, для которых это высказывание ложно. Высказывание
$a\wedge \neg( b \wedge \neg c)$ ложно, когда его отрицание \\ $\neg (a\wedge \neg( b \wedge \neg c))$ истинно. Используя законы Моргана

$$\neg (a\wedge \neg( b \wedge c))=\neg a \vee \neg\neg( b \wedge c)=\neg a \vee ( b \wedge c).$$

Т.е. подходят те слова, для которых верно или $\neg a$, или $( b \wedge c)$.

Поскольку $\neg a$=$\neg$ <<Число четно>> = <<Число нечетно>>, то осталось найти нечетные числа, или те, в которых $7$ цифр и третий разряд чётный. \\

Такими числами являются  1)  $1234567$, 2) $123457$, 3) $2222222$.\\

\section*{Задание 5}

Пусть $A=\{7,5,1,4,2,6,3\}, B= \{ x\ |\ x=2k,\ k\in \mathbb{Z} \}, C=\{0,1,2,3,4,5,6,7,8,9\}$. Для каких $x\in C$ предикат <<$(x \in A)\rightarrow \neg(x \in B)$>> обращается в истину? \\

{\bf Решение:}
Упростим предикат <<$(x \in A)\rightarrow \neg(x \in B)$>> $=$\\ $=a(x)\rightarrow \neg b(x)$, где $a(x)$ и $b(x)$ обозначают предикаты $(x \in A)$ и $(x \in B)$. \\

Согласно формуле (*) из предыдущего задания 
$$a(x)\rightarrow \neg b(x)=\neg (a(x) \wedge b(x)).$$
Предикату $\neg (a(x) \wedge c(x))$ соответствует множество $x \notin (A \cap B$). \\
Итак, для решения задачи нужно найти элементы из $C$, которые не являются элементами множества $A \cap B$, т.е. найти множество $C \setminus (A \cap B)$ .\\
$B$ - множество чётных чисел, $A$ - множество целых чисел от 1 до 7 включительно. Значит,  $A \cap B = \{2,4,6\}$. 
Таким образом, $C \setminus \{2,4,6\} =$ \\ $ =\{0,1,3,5,7,8,9\}$. \\

{\bf Ответ:} Для 0,1,3,5,7,8,9.
\section*{Задание 6}
Докажите, что сумма первых $n$ четных натуральных чисел равняется

$$2+4+6+8+\ldots +2n =n(n+1).$$ 
{\bf Доказательство.} \\
Обозначим равенство через $A_n$ и докажем его по индукции. \\
\begin{itemize}
\item Базис индукции ($A_1$): \\
Докажем утверждение для $n=1$: \\
$$2 = 1(1+1).$$
\item Шаг индукции ($A_{n+1}$): \\
Предположим, что $A_n$ верно, то есть \\
$$2+4+6+8+\ldots +2n =n(n+1).$$ 
Докажем утверждение для $n+1$. Прибавим к обеим частям равенства число $2(n+1)$: \\
$$2+4+6+8+\ldots +2n + 2(n+1) =n(n+1)+2(n+1) = n^2+3n+2 = (n+1)(n+2),$$
то есть $A_{n+1}$. Шаг индукции и всё доказательство завершены.\\


\end{itemize}

\end{document}