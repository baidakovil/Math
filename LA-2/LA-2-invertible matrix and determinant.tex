\documentclass[a4paper,12pt]{article}
\usepackage[utf8]{inputenc}
\usepackage{graphicx}
\usepackage{amssymb}
\usepackage{amsmath}
\usepackage[T2A]{fontenc}
\usepackage[russian]{babel}
\pagenumbering{gobble}
\usepackage{wrapfig}
\usepackage{color}


\begin{document}

\title{<<Линейная алгебра:\\ Обратная матрица. Определитель>>\\ \vspace{12pt} Домашнее задание}
\author{Байдаков Илья}
\date{\today}
\maketitle

\section*{Задание 1}
Найдите матрицу $A^{-1}$, где

$$
A=\begin{pmatrix}
1&1&0\\
0&1&0\\
0&3&3
\end{pmatrix}
$$
\noindent{\bf Решение.} Воспользуемся методом Жордана—Гаусса:
$$ 
\left(
\begin{array}{ccc|ccc}
1&1&0&1&0&0\\
0&1&0&0&1&0\\
0&3&3&0&0&1
\end{array}
\right)
\overset{I - II}{\overset{III - II \cdot 3}{\longmapsto}}
\left(
\begin{array}{ccc|ccc}
1&0&0&1&-1&0\\
0&1&0&0&1&0\\
0&0&3&0&-3&1
\end{array}
\right)
\longmapsto
$$
$$ 
\overset{III \cdot \frac{1}{3}}{\longmapsto}
\left(
\begin{array}{ccc|ccc}
1&0&0&1&-1&0\\
0&1&0&0&1&0\\
0&0&1&0&-1&\frac{1}{3}
\end{array}
\right)
$$
\noindent{\bf Ответ.}
$A^{-1}= 
\begin{pmatrix}
1&-1&0\\
0&1&0\\
0&-1&\frac{1}{3}
\end{pmatrix}
$
\newpage
\section*{Задание 2}
Найдите матрицу $X$, удовлетворяющую равенству

$$
\begin{pmatrix}
0&0&1\\
0&1&0\\
1&0&0
\end{pmatrix}
\cdot
X
\cdot
\begin{pmatrix}
1&1&1\\
1&2&3\\
1&4&9
\end{pmatrix}
=
\begin{pmatrix}
1&2&3\\
2&4&6\\
3&6&9
\end{pmatrix}
$$

\vspace{5pt}

\noindent{\bf Решение.} Известно: если $P$ и $Q$ --- квадратные обратимые матрицы, то 
$$PXQ=S \qquad \Leftrightarrow \qquad X=P^{-1}SQ^{-1}.$$ 
Введём обозначения:
$$
P=\begin{pmatrix}
0&0&1\\
0&1&0\\
1&0&0
\end{pmatrix},
Q=
\begin{pmatrix}
1&1&1\\
1&2&3\\
1&4&9
\end{pmatrix},
S=
\begin{pmatrix}
1&2&3\\
2&4&6\\
3&6&9
\end{pmatrix}
$$
Найдём $P^{-1}$: \par
$$ 
\left(
\begin{array}{ccc|ccc}
0&0&1&1&0&0\\
0&1&0&0&1&0\\
1&0&0&0&0&1
\end{array}
\right)
\overset{I \Leftrightarrow III}{\longmapsto}
\left(
\begin{array}{ccc|ccc}
1&0&0&0&0&1\\
0&1&0&0&1&0\\
0&0&1&1&0&0
\end{array}
\right)
\longmapsto
P^{-1}=
\left(
\begin{array}{ccc}
0&0&1\\
0&1&0\\
1&0&0
\end{array}
\right).
$$
Найдём $Q^{-1}$: \par
$$\left(
\begin{array}{ccc|ccc}
1&1&1&1&0&0\\
1&2&3&0&1&0\\
1&4&9&0&0&1
\end{array}
\right)
\overset{II - I}{\overset{III - I}{\longmapsto}}
\left(
\begin{array}{ccc|ccc}
1&1&1&1&0&0\\
0&1&2&-1&1&0\\
0&3&8&-1&0&1
\end{array}
\right)
\longmapsto
$$
$$ 
\overset{III - 3\cdot II}{\longmapsto}
\left(
\begin{array}{ccc|ccc}
1&1&1&1&0&0\\
0&1&2&-1&1&0\\
0&0&2&2&-3&1
\end{array}
\right)
\overset{III\cdot \frac{1}{2}}{\longmapsto}
\left(
\begin{array}{ccc|ccc}
1&1&1&1&0&0\\
0&1&2&-1&1&0\\
0&0&1&1& -\frac{3}{2}&\frac{1}{2}
\end{array}
\right)
\longmapsto
$$

$$ 
\overset{II - 2\cdot III}{\longmapsto}
\left(
\begin{array}{ccc|ccc}
1&1&1&1&0&0\\
0&1&0&-3&4&-1\\
0&0&1&1& -\frac{3}{2}&\frac{1}{2}
\end{array}
\right)
\overset{I - II - III}{\longmapsto}
\left(
\begin{array}{ccc|ccc}
1&0&0&3&-\frac{5}{2}&\frac{1}{2}\\
0&1&0&-3&4&-1\\
0&0&1&1& -\frac{3}{2}&\frac{1}{2}
\end{array}
\right)
\longmapsto
$$
$$
\longmapsto
Q^{-1}=
\left(
\begin{array}{ccc}
3&-\frac{5}{2}&\frac{1}{2}\\
-3&4&-1\\
1& -\frac{3}{2}&\frac{1}{2}
\end{array}
\right).
$$
Посчитаем $X$:
$$X=P^{-1}SQ^{-1}=
\begin{pmatrix}
0&0&1\\
0&1&0\\
1&0&0
\end{pmatrix}
\cdot
\begin{pmatrix}
1&2&3\\
2&4&6\\
3&6&9
\end{pmatrix}
\cdot
\begin{pmatrix}
3&-\frac{5}{2}&\frac{1}{2}\\
-3&4&-1\\
1& -\frac{3}{2}&\frac{1}{2}
\end{pmatrix}
=
\begin{pmatrix}
0&3&0\\
0&2&0\\
0&1&0
\end{pmatrix}.
$$
\noindent{\bf Ответ.}
$
X=\begin{pmatrix}
0&3&0\\
0&2&0\\
0&1&0
\end{pmatrix}.
$
\section*{Задание 3}
Найдите определитель матрицы $A$
$$
\begin{pmatrix}
1&2&3\\
5&1&4\\
3&2&5
\end{pmatrix}
$$
\noindent{\bf Решение.} Приведём матрицу к треугольному виду:
$$
\begin{pmatrix}
1&2&3\\
5&1&4\\
3&2&5
\end{pmatrix}
\overset{III- I \cdot 3}{\overset{II- I \cdot 5}{\longmapsto}}
\begin{pmatrix}
1&2&3\\
0&-9&-11\\
0&-4&-4
\end{pmatrix}
\overset{III- II \cdot \frac{4}{9}}{\longmapsto}
\begin{pmatrix}
1&2&3\\
0&-9&-11\\
0&0&\frac{8}{9}
\end{pmatrix}
$$
Тогда определитель будет равен произведению элементов на главной диагонали:
$$det(A)=1 \cdot -9 \cdot  \frac{8}{9} = -8$$
\noindent{\bf Ответ.} Определитель равен $-8$.
\section*{Задание 4}
Используя общее определение для определителя $n \times n$, вычислите следующий определитель:
\[ 
\begin{vmatrix}
x & y & 0 & 0\\
0 & x & y & 0\\
0 & 0 & x & y\\
y & 0 & 0 & x
\end{vmatrix}
\]
\noindent{\bf Решение.} Посчитаем отдельно все слагаемые $S$ определителя, множители которых не содержат нулевых элементов:
\begin{itemize}
\item 
$$
\begin{vmatrix}
\textcolor{red}{x} & y & 0 & 0\\
0 & \textcolor{red}{x} & y & 0\\
0 & 0 & \textcolor{red}{x} & y\\
y & 0 & 0 & \textcolor{red}{x}
\end{vmatrix}, \qquad
sgn (\sigma) = -1^0 = 1, \qquad
S = x^4$$

\item 
$$
\begin{vmatrix}
x & \textcolor{red}{y} & 0 & 0\\
0 & x & \textcolor{red}{y} & 0\\
0 & 0 & x & \textcolor{red}{y}\\
\textcolor{red}{y} & 0 & 0 & x
\end{vmatrix}, \qquad
sgn (\sigma) = -1^3 = -1, \qquad
S = -y^4$$
\end{itemize}
Других вариантов нет, значит, 
\[ 
\begin{vmatrix}
x & y & 0 & 0\\
0 & x & y & 0\\
0 & 0 & x & y\\
y & 0 & 0 & x
\end{vmatrix} = x^4-y^4.
\]
\noindent{\bf Ответ.} Определитель равен $x^4-y^4$.
\section*{Задание 5}
С помощью приведения матрицы к треугольному виду вычислите определитель размера $n \times n$, где $n \geqslant 2$:
\[
\begin{vmatrix}
4 & 3 & \dots & 3 & 3\\
3 & 4 & \dots & 3 & 3\\
\hdotsfor{5}\\
3 & 3 & \dots & 4 & 3\\
3 & 3 & \dots & 3 & 4
\end{vmatrix}
\]
Дайте ответ в зависимости от $n$. \par
\noindent{\bf Решение.}
Определитель матрицы треугольного вида равен произведению элементов на главной диагонали матрицы. Проследим, что происходит с этими элементами при преобразовании к нижней треугольной матрице, начиная с нижней строки .\par
Для $n=2$:
$$
\begin{pmatrix}
4 & 3\\
3 & 4\\
\end{pmatrix}
\overset{I-II}{\longmapsto}
\begin{pmatrix}
1 & -1\\
3 & 4\\
\end{pmatrix}
\overset{I+II \cdot \frac{1}{4}}{\longmapsto}
\begin{pmatrix}
\frac{7}{4} & 0\\
3 & 4\\
\end{pmatrix}
$$
\par
Для $n=3$:
$$
\begin{pmatrix}
4 & 3 & 3\\
3 & 4 & 3\\
3 & 3 & 4\\
\end{pmatrix}
\overset{I-II}{\overset{(II-III) + III \cdot \frac{1}{4}}{\longmapsto}}
\begin{pmatrix}
1 & -1 & 0\\
\frac{3}{4} & \frac{7}{4} & 0 \\
3 & 3 & 4\\
\end{pmatrix}
\overset{I+II \cdot \frac{4}{7}}{\longmapsto}
\begin{pmatrix}
\frac{10}{7} & 0 & 0\\
\frac{3}{4} & \frac{7}{4} & 0 \\
3 & 3 & 4\\
\end{pmatrix}
$$
\par
Для $n=4$:
$$
\begin{pmatrix}
4 & 3 & 3 & 3\\
3 & 4 & 3 & 3\\
3 & 3 & 4 & 3\\
3 & 3 & 4 & 4\\
\end{pmatrix}
\overset{I-II}{\overset{II-III}{\overset{(III-IV) + IV \cdot \frac{1}{4}}{\longmapsto}}}
\begin{pmatrix}
1 & -1 & 0 & 0\\
0 & 1 & -1 & 0\\
\frac{3}{4} & \frac{3}{4} & \frac{7}{4} & 0\\
3 & 3 & 4 & 4\\
\end{pmatrix}
\longmapsto
$$
$$
\overset{II+III \cdot \frac{4}{7}}{\longmapsto}
\begin{pmatrix}
1 & -1 & 0 & 0\\
\frac{3}{7} & \frac{10}{7} & 0 & 0\\
\frac{3}{4} & \frac{3}{4} & \frac{7}{4} & 0\\
3 & 3 & 4 & 4\\
\end{pmatrix}
\overset{I+II \cdot \frac{7}{10}}{\longmapsto}
\begin{pmatrix}
\frac{13}{10} & 0 & 0 & 0\\
\frac{3}{7} & \frac{10}{7} & 0 & 0\\
\frac{3}{4} & \frac{3}{4} & \frac{7}{4} & 0\\
3 & 3 & 4 & 4\\
\end{pmatrix}
$$
Сделаем наблюдения:
\begin{itemize}
\item Преобразования строк одинаковы для $n$-й, $(n-1)$-й и т.д. строки для любой матрицы $n \times n$
\item Значения элементов на главной диагонали и слева от неё одинаковы для $n$-й, $(n-1)$-й и т.д. строки для любой матрицы $n \times n$
\item Элемент $a_{1,1}$ всегда есть сумма $1$ и произведения $a_{2,1} \times \frac{1}{a_{2,2}}$
\end{itemize}
Учитывая это, утверждаем, что каждый элемент на главной диагонали может быть выражен только через $n$ и значения элементов на $n$-й строке матрицы. Все эти элементы составляют последовательность чисел $S_n$ для $n \geq 2$:
$$S_n=\frac{7}{4},\frac{10}{7},\frac{13}{10},\frac{16}{13}, \ldots$$
каждый член которой можно выразить через $n$:
$$S_n=\frac{3n + 1}{3n - 2}.$$
Тогда определитель матрицы $n \times n$ записывается в виде:
$$det=4 \cdot \prod_2^{n}\frac{3n + 1}{3n - 2}.$$
\noindent{\bf Ответ.} Определитель равен $4 \cdot \prod_2^{n}\frac{3n + 1}{3n - 2}$.

\section*{Задание 6}
Используя формулы Крамера, решите следующую систему линейных уравнений:
\[
\left\{
\begin{aligned}
-&x_1 + \frac{x_2}{2} + 3 x_3 = 2\\
-&2x_1 + 3x_2 + 11x_3 = 2\\
&4x_1 -2x_2 -11x_3 = 2\\
\end{aligned}
\right.
\]
\noindent{\bf Решение.} \par
Имеем
$$A=
\begin{pmatrix}
-1 & {1}/2 & 3\\
-2 & 3 & 11\\
4 & -2 & -11
\end{pmatrix},
B_1=
\begin{pmatrix}
2 & {1}/2 & 3\\
2 & 3 & 11\\
2 & -2 & -11
\end{pmatrix},
$$
$$
B_2=
\begin{pmatrix}
-1 & 2 & 3\\
-2 & 2 & 11\\
4 & 2 & -11
\end{pmatrix},
B_3=
\begin{pmatrix}
-1 & {1}/2 & 2\\
-2 & 3 & 2\\
4 & -2 & 2
\end{pmatrix}
$$
\par
Отсюда $\det A = -2, \det B_1=-30, \det B_2=52, \det B_3=-20$.\par
Далее $x_1 = \frac{-30}{-2}=15,x_2=\frac{52}{-2}=-26,x3=\frac{-20}{-2}=10$. \par
\noindent{\bf Ответ. }$x_1 =15,x_2=-26,x3=10$.

\section*{Задание 7}

Пусть $A\in M_n(\mathbb{R})$ такова, что $A^m = 0$ для некоторого натурального $m$. Покажите, что матрица $(E - A)$ обратима ({\it указание: найдите явный вид обратной матрицы}).\par
\noindent{\bf Решение.} \par

Заметим, что при умножении матрицы $(E - A)$ на матрицу вида\\$(E+A^1+A^2+\ldots+A^{m-1})$ матрица $A$, входящая в оба множителя,сокращается и произведением является $E$:
$$(E-A)\cdot(E+A^1+A^2+\cdot+A^{m-1})=$$
$$
=E^2+EA^1+EA^2+...+EA^{m-1}-AE-A^2-A^3-...-A^{m-1}-A^m=
$$
$$=E-A^m=E$$\par
По определению, матрица $A\in M_n(\mathbb{R})$ обратима, если существует\\$R\in M_n(\mathbb{R})$ такая, что $AR=E$, что и было показано.\par
Итак, явный вид матрицы обратной матрицы:
$$(E-A)^{-1}=E+\sum_{k=1}^{m-1}A^k.$$
\end{document}

