\documentclass[a4paper,12pt]{article}
\usepackage[utf8]{inputenc}
\usepackage{graphicx}
\usepackage{amssymb}
\usepackage{amsmath}
\usepackage[T2A]{fontenc}
\usepackage[russian]{babel}
\pagenumbering{gobble}
\usepackage{wrapfig}
\usepackage{color}



\begin{document}

\title{<<Линейная алгебра:\\ Векторные пространства, базис, ранг>>\\ \vspace{12pt} Домашнее задание №3}
\author{Байдаков Илья}
\date{\today}
\maketitle

\section*{Задание 1}
Используя формулы разложения определителя по строке/столбцу, вычислите следующий определитель:

$$
\det\begin{pmatrix}
a_0&-1&0&0&0\\
a_1&t&-1&0&0\\
a_2&0&t&-1&0\\
a_3&0&0&t&-1\\
a_4&0&0&0&t
\end{pmatrix}
$$
Запишите ответ в виде многочлена от $t$.
\vspace{5pt} \\
\noindent{\bf Решение.} \\ Воспользуемся формулой разложения по столбцу 1:
$$\det=a_0 \cdot A_{1,1}+a_1 \cdot A_{1,2}+a_2 \cdot A_{1,3}+a_3 \cdot A_{14}+a_4 \cdot A_{1,5} \qquad (*)$$
Посчитаем алгебраические дополнения:
$$A_{1,1}=(-1)^{2}
\begin{vmatrix}
t&-1&0&0\\
0&t&-1&0\\
0&0&t&-1\\
0&0&0&t
\end{vmatrix}
=t^4,$$
$$A_{1,2}=(-1)^{3}
\begin{vmatrix}
-1&0&0&0\\
0&t&-1&0\\
0&0&t&-1\\
0&0&0&t
\end{vmatrix}
=t^3,$$
$$A_{1,3}=(-1)^{4}
\begin{vmatrix}
-1&0&0&0\\
t&-1&0&0\\
0&0&t&-1\\
0&0&0&t
\end{vmatrix}
= \det 
\begin{vmatrix}
-1&0\\
t&-1\\
\end{vmatrix}
\cdot 
\det 
\begin{vmatrix}
t&-1\\
0&t
\end{vmatrix}
=1 \cdot t^2 = t^2$$
$$A_{1,4}=(-1)^{5}
\begin{vmatrix}
-1&0&0&0\\
t&-1&0&0\\
0&t&-1&0\\
0&0&0&t
\end{vmatrix}
=t,$$
$$A_{1,5}=(-1)^{6}
\begin{vmatrix}
-1&0&0&0\\
t&-1&0&0\\
0&t&-1&0\\
0&0&t&-1\\
\end{vmatrix}
=1$$
Получаем ответ по формуле (*).
\vspace{5pt}\\
\noindent{\bf Ответ. $\det = f(t)=a_0 \cdot t^4+a_1 \cdot t^3 + a_2 \cdot t^2+a_3 \cdot t+a_4$}

\section*{Задание 2}
 а) Пусть $V = M_2(\mathbb{R})$, $U = \left\{\left(\begin{smallmatrix}
a & b \\
c & d
\end{smallmatrix} \right) \mid a + d = 0 \right\}$. Покажите, что $U$ -- подпространство $V$. Иными словами, нужно показать, что множество $2\times 2$ матриц с нулевым следом является подпространством пространства всех вещественных $2\times 2$ матриц ({\it следом} матрицы называется сумма элементов главной диагонали матрицы).
Найдите размерность $U$ и предъявите некоторый базис $U$.

б) В выбранном базисе найдите координаты матрицы $\left(\begin{smallmatrix}
-1 & 2 \\
3 & 1
\end{smallmatrix} \right)$ (то есть укажите, как эта матрица выражается через базисные). 
\vspace{5pt} \\
\noindent{\bf Решение.}\\
а) Если $U$ - подпространство $V$, то выполняются следующие условия: 
\begin{itemize}
\item Если $u$, $v \in U$, то и $u+v \in U$. \\
В нашем случае, для произвольных матриц $u= \left(\begin{smallmatrix}
a & b \\
c & d
\end{smallmatrix}\right)$ и $v= \left(\begin{smallmatrix}
e & f \\
g & h
\end{smallmatrix}\right)$ из $U$: матрица 
$$x=u+v=\left(\begin{smallmatrix}
a+e & b+f \\
c+g & d+h
\end{smallmatrix}\right) \in U, $$
потому что:
$$\left\{
\begin{aligned}
a+d=0 \\
e+h=0 
\end{aligned}
\right.
\longrightarrow (a+e)+(d+h)=0 $$
\item Если $u \in U$,и $r \in \mathbb{R}$ то $r \cdot u \in U$. \\
В нашем случае, для произвольной матрицы $u= \left(\begin{smallmatrix}
a & b \\
c & d
\end{smallmatrix}\right)$ из $U$ и $r \in \mathbb{R}$: 
$$r \cdot u = \begin{pmatrix}
r \cdot a & r \cdot b \\
r \cdot c & r \cdot d
\end{pmatrix}\in U,$$ потому что
$$a+d =0 \longrightarrow r \cdot a + r \cdot d = 0$$
\end{itemize}
Поскольку эти два условия выполняются, то $U$ -- подпространство $V$.\par
Пример базиса $U$:
$$S_1=\begin{pmatrix}
1 & 0 \\
0 & -1
\end{pmatrix},
S_2=\begin{pmatrix}
0 & 1 \\
0 & 0
\end{pmatrix},
S_3=\begin{pmatrix}
0 & 0 \\
1 & 0
\end{pmatrix}
$$
Размерность $U$ равна числу элементов в его базисе, т.е. $\dim U=3$ \\
б)  Матрица $\left(\begin{smallmatrix}
-1 & 2 \\
3 & 1
\end{smallmatrix} \right)$ через указанный выше базис выражается так:
$$\begin{pmatrix}
-1 & 2 \\
3 & 1
\end{pmatrix}=S_1 \cdot (-1) + S_2 \cdot 2 + S_3 \cdot 3.$$

\section*{Задание 3}
Найдите все значения $\lambda\in \mathbb R$, при которых вектор $v$ линейно выражается через векторы $a_1$, $a_2$, $a_3$, где
$$a_1 = 
\begin{pmatrix}
{2}\\{3}\\{5}
\end{pmatrix},\;
a_2 = 
\begin{pmatrix}
{3}\\{7}\\{8}
\end{pmatrix},\;
a_3 = 
\begin{pmatrix}
{1}\\{-6}\\{1}
\end{pmatrix},\;
v = 
\begin{pmatrix}
{7}\\{-2}\\{\lambda}
\end{pmatrix}.$$
\noindent{\bf Решение.}\par
Вектор $v$ линейно выражается через векторы $a_1$, $a_2$, $a_3$, когда у этой системы есть решения:
$$
\left\{
\begin{aligned}
&2x_1 + 3x_2 + x_3 = 7\\
&3x_1 + 7x_2 + -6x_3 = -2\\
&5x_1 + 8x_2 + x_3 = \lambda
\end{aligned}
\right.
\qquad (*)$$
Перепишем в матричном виде:
$$\begin{pmatrix}
2 & 3 & 1\\
3 & 7 & -6\\
5 & 8& 1
\end{pmatrix} \cdot
\begin{pmatrix}
{x_1}\\{x_2}\\{x_3}
\end{pmatrix}
 = 
\begin{pmatrix}
{7}\\{-2}\\{\lambda}
\end{pmatrix}
$$
Решим систему для известных значений вектора $v$:
$$ 
\left(\begin{array}{ccc|c}
2 & 3 & 1 & 7\\
3 & 7 & -6 & -2\\
\end{array} \right)
\overset{II-I}{\longmapsto}
\left(\begin{array}{ccc|c}
2 & 3 & 1 & 7\\
1 & 4 & -7 & -9\\
\end{array} \right)
\longmapsto
$$
$$
\overset{I-II}{\longmapsto}
\left(\begin{array}{ccc|c}
1 & -1 & 8 & 16\\
1 & 4 & -7 & -9\\
\end{array} \right)
\overset{II-I}{\longmapsto}
\left(\begin{array}{ccc|c}
1 & -1 & 8 & 16\\
0 & 5 & -15 & -25\\
\end{array} \right)
\longmapsto
$$
$$
\overset{I+II \cdot \frac{1}{5}}{\longmapsto}
\left(\begin{array}{ccc|c}
1 & 0 & 5 & 11\\
0 & 5 & -15 & -25\\
\end{array} \right)
\overset{II \cdot \frac{1}{5}}{\longmapsto}
\left(\begin{array}{ccc|c}
1 & 0 & 5 & 11\\
0 & 1 & -3 & -5\\
\end{array} \right)
$$
Имеем $x_1$ и $x_2$ главными переменными, а $x_3$ --- свободной. Полученная матрица соответствует системе уравнений:
$$
\left\{
\begin{aligned}
&x_1 + 5 x_3 = 11\\
&x_2 - 3 x_3 = -5\\
\end{aligned}
\right.
\Leftrightarrow
\left\{
\begin{aligned}
&x_1 = 11-5 x_3\\
&x_2 = -5 +3 x_3\\
\end{aligned}
\right., \qquad x_3 \in \mathbb{R}.
$$
При таких значениях  $x_1$, $x_2$ и $x_3$ вектор $v$ линейно выражается через $a_1$, $a_2$, $a_3$. Посмотрим, какие значения при этом будет принимать $\lambda$, подставив найденные значения в (*):
$$\lambda=5x_1+8x_2+x_3=5(11-5 x_3)+8(-5 +3 x_3)+x_3=$$
$$=55-25x_3-40+24x_3+x_3=15$$
Таким образом, вектор $v$ линейно выражается через векторы $a_1$, $a_2$, $a_3$ при $\lambda=15$.\\
\noindent{\bf Ответ.} $\lambda=15$.

\section*{Задание 4}
Найдите какой-нибудь базис системы векторов и выразите через него все остальные векторы системы:\\
$a_1 = (2, -1, 3, 5), \ a_2 = (4, -3, 1, 3), \ a_3 = (3, -2, 3, 4)$ ,\\
$ \ a_4 = (4, -1, 15, 17), \ a_5 = (7, -6, -7, 0).$ \\
\noindent{\bf Решение.}\\
Запишем векторы по строкам в матрицу:
$$A=\begin{pmatrix}
2 & 4 & 3 & 4 & 7\\
-1 & -3 & -2 & -1 & -6\\
3 & 1 & 3 & 15 & -7\\
5 & 3 & 4 & 17 &0 \\
\end{pmatrix}$$ \par
Приведём её элементарными преобразованиями строк к каноническому ступенчатому виду:
$$A=\begin{pmatrix}
2 & 4 & 3 & 4 & 7\\
-1 & -3 & -2 & -1 & -6\\
3 & 1 & 3 & 15 & -7\\
5 & 3 & 4 & 17 &0 \\
\end{pmatrix}
\longrightarrow A'=
\begin{pmatrix}
1 & 0 & 0 & 2 & 1\\
0 & 1 & 0 & -3 & 5\\
0 & 0 & 1 & 4 & -5\\
0 & 0 & 0 & 0 &0 \\
\end{pmatrix}
$$
\par
Поскольку при преобразовании к ступенчатому виду одна из координат оказалась <<утеряна>>, то в качестве базисных возьмём те вектора, которые соответствуют главным координатам матрицы $A'$ в исходной матрице $A$. \par
Итак, базис исходной системы векторов: 
$$a_1 =
\begin{pmatrix}
2 \\
-1 \\
3 \\
5 \\
\end{pmatrix},
a_2 = 
\begin{pmatrix}
4 \\
-3 \\
1 \\
3 \\
\end{pmatrix},
a_3 = 
\begin{pmatrix}
3 \\
-2 \\
3 \\
4 \\
\end{pmatrix}.$$
Выразим через этот базис все остальные векторы системы:
$$2a_1-3a_2+4a_3=
\begin{pmatrix}
4 \\
-1 \\
15 \\
17 \\
\end{pmatrix}=a_4,$$
и последний
$$a_1+5a_2-5a_3=
\begin{pmatrix}
7 \\
-6 \\
-7 \\
0 \\
\end{pmatrix}=a_5.$$
\newpage
\section*{Задание 5}
Для каждого значения $\lambda\in \mathbb R$ найдите ранг матрицы
$$A=
\begin{pmatrix}
{-\lambda}&{1}&{2}&{3}&{1}\\
{1}&{-\lambda}&{3}&{2}&{1}\\
{2}&{3}&{-\lambda}&{1}&{1}\\
{3}&{2}&{1}&{-\lambda}&{1}\\
\end{pmatrix}.$$\par
\noindent{\bf Решение.}\par
\noindent Рассмотрим подматрицу $B$ максимального размера:
$$B=
\begin{pmatrix}
{1}&{2}&{3}&1\\
{-\lambda}&{3}&{2}&1\\
{3}&{-\lambda}&{1}&1\\
{2}&{1}&{-\lambda}&1\\
\end{pmatrix}.$$ \par
Если $B$ обратима для некоторых $\lambda$, значит, для этих $\lambda$ минорный ранг матрицы $A$ будет равен максимально возможному, т.е. $4$.
\par
Приведём $B$ к треугольному виду:
$$
\begin{pmatrix}
{1}&{2}&{3}&1\\
{-\lambda}&{3}&{2}&1\\
{3}&{-\lambda}&{1}&1\\
{2}&{1}&{-\lambda}&1\\
\end{pmatrix}
\overset{III-I}{\longmapsto}
\begin{pmatrix}
{1}&{2}&{3}&1\\
{-\lambda}&{3}&{2}&1\\
{2}&{-\lambda-2}&{-2}&0\\
{2}&{1}&{-\lambda}&1\\
\end{pmatrix}
\longmapsto
$$
$$
\overset{II-I}{\longmapsto}
\begin{pmatrix}
{1}&{2}&{3}&1\\
{-\lambda-1}&{1}&{-1}&0\\
{2}&{-\lambda-2}&{-2}&0\\
{2}&{1}&{-\lambda}&1\\
\end{pmatrix}
\overset{I-IV}{\longmapsto}
\begin{pmatrix}
{-1}&{1}&{3+\lambda}&0\\
{-\lambda-1}&{1}&{-1}&0\\
{2}&{-\lambda-2}&{-2}&0\\
{2}&{1}&{-\lambda}&1\\
\end{pmatrix}
\longmapsto
$$
$$
\overset{II-III \cdot \frac{1}{2}}{\longmapsto}
\begin{pmatrix}
{-1}&{1}&{3+\lambda}&0\\
{-\lambda-2}&{\frac{1}{2}\lambda+2}&{0}&0\\
{2}&{-\lambda-2}&{-2}&0\\
{2}&{1}&{-\lambda}&1\\
\end{pmatrix}
\overset{\overset{II \cdot \frac{2}{(\lambda+4)}}{\longmapsto}}{\lambda \neq -4}
\begin{pmatrix}
{-1}&{1}&{3+\lambda}&0\\
{-\frac{2(\lambda+2)}{(\lambda+4)}}&{1}&{0}&0\\
{2}&{-\lambda-2}&{-2}&0\\
{2}&{1}&{-\lambda}&1\\
\end{pmatrix}
\longmapsto
$$
$$
\overset{I-II}{\longmapsto}
\begin{pmatrix}
{\frac{2(\lambda+2)}{(\lambda+4)}-1}&{0}&{3+\lambda}&0\\
{-\frac{2(\lambda+2)}{(\lambda+4)}}&{1}&{0}&0\\
{2}&{-(\lambda+2)}&{-2}&0\\
{2}&{1}&{-\lambda}&1\\
\end{pmatrix}
\overset{\overset{III \cdot \frac{1}{(\lambda+2)}}{\longmapsto}}{\lambda \neq -2}
\begin{pmatrix}
{\frac{2(\lambda+2)}{(\lambda+4)}-1}&{0}&{3+\lambda}&0\\
{-\frac{2(\lambda+2)}{(\lambda+4)}}&{1}&{0}&0\\
{\frac{2}{(\lambda+2)}}&{-1}&{-\frac{2}{(\lambda+2)}}&0\\
{2}&{1}&{-\lambda}&1\\
\end{pmatrix}
\longmapsto
$$
$$
\overset{III+II}{\longmapsto}
\begin{pmatrix}
{\frac{2(\lambda+2)}{(\lambda+4)}-1}&{0}&{3+\lambda}&0\\
{-\frac{2(\lambda+2)}{(\lambda+4)}}&{1}&{0}&0\\
{\frac{2}{(\lambda+2)}-\frac{2(\lambda+2)}{(\lambda+4)}}&{0}&{-\frac{2}{(\lambda+2)}}&0\\
{2}&{1}&{-\lambda}&1\\
\end{pmatrix}
\overset{III \cdot (-\frac{(\lambda+2)}{2})}{\longmapsto}
\begin{pmatrix}
{\frac{2(\lambda+2)}{(\lambda+4)}-1}&{0}&{3+\lambda}&0\\
{-\frac{2(\lambda+2)}{(\lambda+4)}}&{1}&{0}&0\\
{\frac{(\lambda+2)^2}{(\lambda+4)}-1}&{0}&{1}&0\\
{2}&{1}&{-\lambda}&1\\
\end{pmatrix}
\longmapsto
$$
$$
\overset{I-III \cdot (\lambda+3)}{\longmapsto}
\begin{pmatrix}
{\frac{2(\lambda+2)}{(\lambda+4)}-1+(\lambda+3)-\frac{(\lambda+2)^2}{(\lambda+4)}\cdot (\lambda+3)}&{0}&{0}&0\\
{-\frac{2(\lambda+2)}{(\lambda+4)}}&{1}&{0}&0\\
{\frac{(\lambda+2)^2}{(\lambda+4)}-1}&{0}&{1}&0\\
{2}&{1}&{-\lambda}&1\\
\end{pmatrix}
$$\par
Мы привели подматрицу $B$ к треугольному виду, в котором только один элемент на главной диагонали зависит от $\lambda$, при этом в выводе не допускалось, что $\lambda = -4$ и $\lambda = -2$.
Если $a_{11}$ этой подматрицы не равен нулю для каких-либо $\lambda$, то для этих $\lambda$ выполняется: $\det B \neq 0 \longrightarrow$ \\ 
$\longrightarrow \text{rk }B = 4 \longrightarrow \text{rk }A = 4$. \par
Решим уравнение $a_{11}=0$:
$$\frac{2(\lambda+2)}{(\lambda+4)}-1+(3+\lambda)-\frac{(\lambda+2)^2}{(\lambda+4)}\cdot (3+\lambda)=0
\quad \Leftrightarrow$$
$$\Leftrightarrow \quad
\frac{2(\lambda+2)-(\lambda+2)^2(3+\lambda)}{(\lambda+4)}+\lambda+2=0 
\quad \Leftrightarrow$$
$$\Leftrightarrow \quad
\frac{2(\lambda+2)-(\lambda+2)^2(3+\lambda)+(\lambda+2)(\lambda+4)}{(\lambda+4)}=0 \quad \longrightarrow$$
$$ \longrightarrow \quad
2(\lambda+2)-(\lambda+2)^2(3+\lambda)+(\lambda+2)(\lambda+4)=0
\quad \Leftrightarrow$$
$$ \Leftrightarrow \quad
(\lambda+2)[2-(\lambda+2)(3+\lambda)+(\lambda+4)]=0
\quad \Leftrightarrow$$
$$ \Leftrightarrow \quad
(\lambda+2)(-\lambda^2-4\lambda)=0
\quad \longrightarrow \quad
\left[
\begin{aligned}
&\lambda +2 = 0 \\
&-\lambda(\lambda+4)=0
\end{aligned}
\right. \quad \longrightarrow
$$
$$
\quad \longrightarrow
\left[
\begin{aligned}
&\lambda = -2 \\
&\lambda = 0 \\
&\lambda = -4
\end{aligned}
\right.
.$$ \par
Так мы доказали, что ранг матрицы $A$ равен $4$ для всех значений $\lambda \in \mathbb{R}$, кроме $\lambda = 0$, $\lambda = -2$ и $\lambda = - 4$, для которых ранг пока неизвестен.\par
Для оставшихся значений посчитаем ранг матрицы $A$ приведением к ступенчатому виду: \par
Для $\lambda=0$:
$$A_{\lambda=0}=
\begin{pmatrix}
{0}&{1}&{2}&{3}&{1}\\
{1}&{0}&{3}&{2}&{1}\\
{2}&{3}&{0}&{1}&{1}\\
{3}&{2}&{1}&{0}&{1}\\
\end{pmatrix}
\longmapsto
\begin{pmatrix}
{1}&{0}&{0}&{-1}&{0}\\
{0}&{1}&{0}&{1}&{\frac{1}{3}}\\
{0}&{0}&{1}&{1}&{\frac{1}{3}}\\
{0}&{0}&{0}&{0}&{0}\\
\end{pmatrix}
\longmapsto
\text{rk } A_{\lambda=0} = 3.
$$
\par
Для $\lambda=-2$:
$$A_{\lambda=-2}=
\begin{pmatrix}
{-2}&{1}&{2}&{3}&{1}\\
{1}&{-2}&{3}&{2}&{1}\\
{2}&{3}&{-2}&{1}&{1}\\
{3}&{2}&{1}&{-2}&{1}\\
\end{pmatrix}
\longmapsto
\begin{pmatrix}
{1}&{0}&{0}&{1}&{\frac{1}{4}}\\
{0}&{1}&{0}&{-1}&{0}\\
{0}&{0}&{1}&{1}&{\frac{1}{4}}\\
{0}&{0}&{0}&{0}&{0}\\
\end{pmatrix}
\longmapsto
\text{rk } A_{\lambda=-2} = 3.
$$
\par
Для $\lambda=-4$:
$$A_{\lambda=-4}=
\begin{pmatrix}
{-4}&{1}&{2}&{3}&{1}\\
{1}&{-4}&{3}&{2}&{1}\\
{2}&{3}&{-4}&{1}&{1}\\
{3}&{2}&{1}&{-4}&{1}\\
\end{pmatrix}
\longmapsto
\begin{pmatrix}
{1}&{0}&{0}&{1}&{\frac{1}{5}}\\
{0}&{1}&{0}&{1}&{\frac{1}{5}}\\
{0}&{0}&{1}&{-1}&{0}\\
{0}&{0}&{0}&{0}&{0}\\
\end{pmatrix}
\longmapsto
\text{rk } A_{\lambda=-4} = 3.
$$
\par

Теперь все значения $\lambda\in \mathbb R$ рассмотрены.\par
\vspace{5pt}
\noindent{\bf Ответ.} $\text{rk }A = 3$ при $\lambda = 0$, $\lambda = -2$ и $\lambda = -4$. Для других $\lambda \in \mathbb{R}$ будет $\text{rk }A = 4$.

\section*{Задание 6}
Известно, что присоединенная к $A$ матрица (то есть транспонированная матрица алгебраических дополнений) равна

\[
\widehat A = 
\begin{pmatrix}
1 & -2 & 1\\
3 & -4 & 3\\
-9 & 10 & -7
\end{pmatrix}.
\]

Чему может равняться матрица $A$?
\par
\vspace{5pt}
\noindent{\bf Решение.}\par
Запишем формулу для явного вида обратной матрицы:
$$A^{-1}=\frac{1}{\det A} \widehat A,$$
отсюда
$$\widehat A =A^{-1} \cdot \det A .$$
Если матрица $\widehat A$ обратима, то будет справедливо равенство:
$$(\widehat A)^{-1}\cdot \widehat A = E 
\quad \longrightarrow \quad
(\widehat A)^{-1} A^{-1} \cdot \det A = E .$$\par
Сравнивая это равенство с равенством $A \cdot A^{-1} = E$, находим, что
$$A = (\widehat A)^{-1} \cdot \det A,$$
т.е. найдя $(\widehat A)^{-1}$, мы будем знать матрицу $A$ с точностью до умножения на константу. Найдём $(\widehat A)^{-1}$:
$$
\left(\begin{array}{ccc|ccc}
1 & -2 & 1 & 1 & 0 & 0\\
3 & -4 & 3 & 0 & 1 & 0\\
-9 & 10 & -7 & 0 & 0 & 1
\end{array} \right)
\overset{III + 3 \cdot II}{\longmapsto}
\left(\begin{array}{ccc|ccc}
1 & -2 & 1 & 1 & 0 & 0\\
3 & -4 & 3 & 0 & 1 & 0\\
0 & -2 & 2 & 0 & 3 & 1
\end{array} \right)
\longmapsto
$$
$$
\overset{I - III}{\longmapsto}
\left(\begin{array}{ccc|ccc}
1 & 0 & -1 & 1 & -3 & -1\\
3 & -4 & 3 & 0 & 1 & 0\\
0 & -2 & 2 & 0 & 3 & 1
\end{array} \right)
\overset{II - 3 \cdot I}{\longmapsto}
\left(\begin{array}{ccc|ccc}
1 & 0 & -1 & 1 & -3 & -1\\
0 & -4 & 6 & -3 & 10 & 3\\
0 & -2 & 2 & 0 & 3 & 1
\end{array} \right)
\longmapsto
$$
$$
\overset{II - 3 \cdot III}{\longmapsto}
\left(\begin{array}{ccc|ccc}
1 & 0 & -1 & 1 & -3 & -1\\
0 & 2 & 0 & -3 & 1 & 0\\
0 & -2 & 2 & 0 & 3 & 1
\end{array} \right)
\overset{II \cdot \frac{1}{2}}{\longmapsto}
\left(\begin{array}{ccc|ccc}
1 & 0 & -1 & 1 & -3 & -1\\
0 & 1 & 0 & -\frac{3}{2} & \frac{1}{2} & 0\\
0 & -2 & 2 & 0 & 3 & 1
\end{array} \right)
\longmapsto
$$
$$
\overset{III + 2 \cdot II}{\longmapsto}
\left(\begin{array}{ccc|ccc}
1 & 0 & -1 & 1 & -3 & -1\\
0 & 1 & 0 & -\frac{3}{2} & \frac{1}{2} & 0\\
0 & 0 & 2 & -3 & 4 & 1
\end{array} \right)
\overset{III \cdot \frac{1}{2}}{\longmapsto}
\left(\begin{array}{ccc|ccc}
1 & 0 & -1 & 1 & -3 & -1\\
0 & 1 & 0 & -\frac{3}{2} & \frac{1}{2} & 0\\
0 & 0 & 1 & -\frac{3}{2} & 2 & \frac{1}{2}
\end{array} \right)
\longmapsto
$$
$$
\overset{I + III}{\longmapsto}
\left(\begin{array}{ccc|ccc}
1 & 0 & 0 & -\frac{1}{2} & -1 & -\frac{1}{2}\\
0 & 1 & 0 & -\frac{3}{2} & \frac{1}{2} & 0\\
0 & 0 & 1 & -\frac{3}{2} & 2 & \frac{1}{2}
\end{array} \right)
\quad
\longmapsto \quad
(\widehat A)^{-1} = 
\begin{pmatrix}
-\frac{1}{2} & -1 & -\frac{1}{2}\\
-\frac{3}{2} & \frac{1}{2} & 0\\
-\frac{3}{2} & 2 & \frac{1}{2}
\end{pmatrix}
$$\par
Таким образом,
$$A = \begin{pmatrix}
-\frac{1}{2} & -1 & -\frac{1}{2}\\
-\frac{3}{2} & \frac{1}{2} & 0\\
-\frac{3}{2} & 2 & \frac{1}{2}
\end{pmatrix}
\cdot \det A.$$\par
Предположим, что у $A$ были целочисленные коэффициенты. В таком случае  $\det A = 2 \cdot n$, где $n \in \mathbb{R}$. Тогда при $n=1$, что соответствует $\det A = 2$, будем иметь первую матрицу-<<претендент>>:
$$A_{run1} = \begin{pmatrix}
- 1 & -2 & -1\\
-3 & 1 & 0\\
-3 & 4 & 1
\end{pmatrix}.$$
Проверим $A_{run1}$ на соответствие условию задачи, составив транспонированную матрицу её алгебраических дополнений:
$$\widehat{A_{run1}} =
\begin{pmatrix}
1 & -2 & 1\\
3 & -4 & 3\\
-9 & 10 & -7
\end{pmatrix}
= \widehat{A}.$$
Равенство выполнилось. Следовательно, исходная матрица действительно может иметь такой вид, как  $A_{run1}$.\par
\vspace{10pt}
\noindent{\bf Ответ.} $A$ может равняться $\left(
\begin{smallmatrix}
- 1 & -2 & -1\\
-3 & 1 & 0\\
-3 & 4 & 1
\end{smallmatrix}
\right)
$.
\end{document}		