\documentclass[a4paper,12pt]{article}
\usepackage[utf8]{inputenc}
\usepackage{graphicx}
\usepackage{amssymb}
\usepackage{amsmath}
\usepackage[T2A]{fontenc}
\usepackage[russian]{babel}
\pagenumbering{gobble}
\usepackage{multicol}
\newcommand{\ssolve}{\par\vspace{5pt}\noindent{\bf Решение. }\par}
\newcommand{\aanswer}{\par\vspace{5pt}\noindent{\bf Ответ. }}
\newcommand{\bpm}{\begin{pmatrix}}
\newcommand{\epm}{\end{pmatrix}}
\newcommand{\bvm}{\begin{vmatrix}}
\newcommand{\evm}{\end{vmatrix}}
\newcommand{\bal}{\left\{\begin{aligned}}
\newcommand{\eal}{\end{aligned}\right.}
\newcommand{\llaq}{\quad \longrightarrow \quad}
\newcommand{\ao}[1]{\overset{#1}{\longmapsto}}
\newcommand{\E}{\mathbb{E}}
\newcommand{\Var}{\textbf{Var}}
\newcommand{\Cov}{\textbf{Cov}}

\begin{document}

\title{<<Теория вероятностей:\\ Математическое ожидание и дисперсия>>\\ \vspace{12pt} Домашнее задание №5}
\author{Байдаков Илья}
\date{\today}
\maketitle

\section*{Задание 1}
Банк внедряет новую систему для анализа кредитной истории клиентов. Согласно статистике
банка, наличие у заемщика просрочек по платежам в прошлом (c.в. $X$) и факт погашения кредита позже назначенного срока (с.в. $Y$ ) имеют распределение, заданное таблицей ниже. \par
\textit{(Т.е. случайная величина X принимает значение 0, если человек вносил платежи вовремя, и 1 иначе. Аналогично, Y принимает значение 0, если клиент вернул кредит вовремя, и 1 в противном случае.)} \par
В зависимости от значения корреляции этих величин банк примет решение о том, как учитывать наличие просрочек по платежам для выдачи кредита. \par
Найдите коэффициент корреляции случайных величин X и Y .
\par
\includegraphics[width=0.4 \textwidth]{1.png}\par
\ssolve
По определению,
$$\rho_{X,Y} = \frac{\textbf{Cov}(X,Y)}{\sqrt{\textbf{Var}X}\sqrt{\textbf{Var}Y}}=\frac{0.3-0.4\cdot0.3}{\sqrt{0.4-0.16}\sqrt{0.3-0.09}}\simeq0.802$$
\aanswer $\rho_{X,Y} \simeq 0.802$.



\section*{Задание 2}
Независимые случайные величины $X,Y\sim U(-1,1)$. Рассмотрим \\ $Z=2X+Y$. \par
а) Найдите $\text{Cov}(Z, Y )$. \par
б) Чему равняется $\rho_{Z,Y}$?
\ssolve
а) По определению, $$\textbf{Cov}(Z,Y)=\E (ZY) - \E(Z)\cdot\E (Y) \qquad (*)$$
Т.к. имеем нормальное распределение, то
$$\E(X)= \E(Y)=\frac{a+b}{2}=0$$
$$\E(Z)=\E(2X+Y)=0$$
Из учебника Черновой,
$$\E(Y^2) = \frac{a^2+ab+b^2}{3}=\frac{1}{3}$$
$$\E(ZY)=\E[(2X+Y)Y]=2\E(X)\E(Y)+\E(Y^2)=\frac{1}{3}$$
Подставляем в (*):
$$\textbf{Cov}(Z,Y)=\frac{1}{3}-0\cdot0=\frac{1}{3}$$
б) вычислим дисперсии для нормального распределения. 
По формуле из учебника Черновой
$$\Var (X)= \Var (Y)=\frac{(b-a)^2}{12}=\frac{1}{3}$$
$$\Var(Z) = \Var(2X+Y)=4\Var(X)+\Var(Y)=4\cdot\frac{1}{3}+\frac{1}{3}=\frac{5}{3}$$
По определению,
$$\rho_{Z,Y}=\frac{\Cov (Z,Y)}{\sqrt{\Var Z}\sqrt{\Var Y}}=\frac{1/3}{\sqrt{5/3}\sqrt{1/3}} \simeq 0.447$$
\aanswer а) $\Cov(Z,Y) = \frac{1}{3}.$ б) $\rho_{Z,Y} \simeq 0.447$.




\section*{Задание 3}
Закон распределения случайного вектора
$(X, Y )$ задан таблицей. \par

\includegraphics[width=0.3 \textwidth]{2.png} \par

а) Какие значения принимает условное математическое ожидание $\mathbb{E}(X|Y )$? 
\textit{(Иными словами, требуется найти значения
$\mathbb{E}(X|Y = 1)$ и \\ $\mathbb{E}(X|Y = 3)$.)}
\par
б) С какой вероятностью условное математическое ожидание $\mathbb{E}(X|Y )$ принимает значение $\mathbb{E}(X|Y =1)$? \par
в) Какая из функций ниже выражает условное математическое ожидание $\mathbb{E}(X|Y )$ через $Y$ ? Ответ обоснуйте. \\
1) $\frac{Y}{2} \qquad$ 2) $\frac{1}{2} + \frac{Y}{2} \qquad$ 3) $Y^2 \qquad$ 4) $2Y-1$.

\ssolve
а) Найдём распределение $P_Y(y)$: $P_Y(Y=1)=0.6$, $P_Y(Y=3)=0.4$. \par
Далее, по определению, \par
$$\E(X|Y=1)=\sum_i x_i\cdot p_{X|Y}(x_i|y)=\sum_i x_i\cdot \frac{p_{X|Y}(x_i,y_0)}{P_Y(y_0)}=$$
$$=0\cdot \frac{0.4}{0.6}+2\cdot \frac{0}{0.6}+3 \cdot \frac{0.2}{0.6}=1.$$
Аналогично находим $\E(X|Y=3)=2$. \par
\vspace{5pt}
б) Рассматривая $\mathbb{E}(X|Y)$ как случайную величину, запишем её таблицу распределения. В ней будет два значения, каждое соответствует одному из двух возможных значений $Y$. Значит, вероятность, что $\mathbb{E}(X|Y)$ примет это значение, соответствует вероятности того, что $Y$ примет соответствующее значение: \par
\vspace{5pt}
\begin{tabular}{|c|c|c|}
\hline
$\mathbb{E}(X|Y)$&1&2 \\
\hline
$P(\mathbb{E}(X|Y))$&0.6&0.4\\
\hline 

\end{tabular} \par
\vspace{5pt}
Значит, $\mathbb{E}(X|Y)$ примет значение $\mathbb{E}(X|Y =1)=1$ с вероятностью $0.6$. \par
\vspace{5pt}
в) Правильный ответ (2), т.е.  $f(Y)=\frac{1}{2} + \frac{Y}{2}$. Согласно расчётам в пункте (а), именно эта функция принимает верные значения $\mathbb{E}(X|Y )$ при $Y=1$ и при $Y=3$.

\aanswer \par а) $\E(X|Y=1)=1$, $\E(X|Y=3)=2$. \par б) $P=0.6$. \par в) (2).
\section*{Задание 4}
Пусть с.в. $X$ и $Y$ независимы и имеют распределения $X \sim N(2, 4)$, \\ $Y \sim Exp(2)$. \par
Чему равняется $\mathbb{E}[(X + Y )^2 \cdot \sin Y |Y ]$?

\ssolve 
Пользуясь свойствами линейности, стабильности и вынесения множителя условного мат. ожидания, а также  учитывая независимость у.в., и также что $\E X=2$ и $\E (X^2|Y) = \E (X^2) = \Var X + (\E X)^2$, получаем:
$$\E[(X+Y)^2\cdot \sin Y|Y]=\sin Y \cdot \E[(X^2 + 2XY + Y^2)|Y]=$$
$$=\sin Y \cdot (\E (X^2|Y)+2Y \E (X|Y) + Y^2)=\sin Y \cdot (\E (X^2|Y)+2Y \E (X) + Y^2)=$$
$$=\sin Y \cdot (\E (X^2|Y)+2Y \E (X) + Y^2)=\sin Y \cdot(4+2^2+4Y+Y^2)=$$
$$=(Y^2+4Y+8)\cdot \sin Y.$$
\aanswer $\E = (Y^2+4Y+8)\cdot \sin Y.$
 
\section*{Задание 5}
Среднее время обработки запроса на некотором сервисе равняется $1$ секунде. \par
а) Какова максимальная возможная вероятность того, что запрос будет обрабатываться не меньше $100$ секунд?
Приведите пример ситуации, в которой достигается эта вероятность. \par
б) Пусть теперь известно, что дисперсия времени обработки запроса равняется $1$. Оцените с помощью неравенства Чебышева вероятность того, что запрос будет обрабатываться не меньше $100$ секунд.

\ssolve
а) Обозначая время запроса за $X$ и считая среднее время запроса мат. ожиданием $X$, получаем согласно неравенству Маркова:
$$P(X \geq 100) = P(X \geq 100 \cdot \E X) \leq \frac{1}{100} \longrightarrow P_{max} = \frac{1}{100}.$$
Пример: 99 запросов обработались за пренебрежимо малое время, а следующий обрабатывался 100 секунд. \par
б) Запишем неравенство Чебышёва:
$$P(X \geq 100) = P(X - \E X \geq 99) \leq \frac{\Var X}{x^2} = \frac{1}{99^2} \simeq 10^{-4}.$$
\aanswer а) $P=1/100$. б) $P \simeq 10^{-4}.$

\section*{Задание 6}
Известно, что среднее время решения этой задачи составляет 40 минут. Причем вероятность того, что на решение уйдет не больше $30$ минут, равна $\frac{1}{2}$. Среднее время решения
этой задачи для тех, кто уложился в $30$ минут, равняется $20$ минутам. \par
Каково среднее время решение этой задачи для тех, кто решал ее дольше $30$ минут?

\ssolve
Обозначим время решения задачи $X$. \par
Тогда $\E X = 40$, $P(X \leq 30)=\frac{1}{2}$, $P(X>30)=\frac{1}{2}$ и $\E(X|X \leq 30) = 20$.
Запишем формулу полной вероятности для условного математического ожидания $\E X$:
$$\E X = \E(X|X \leq 30) \cdot P(X \leq 30) + \mathbb{E}(X|X > 30) \cdot P(X>30),$$
и выразим отсюда искомую величину
$$\mathbb{E}(X|X > 30) = (40-10) \cdot 2 = 60 \text{ мин.}$$
\aanswer 60 мин.

\section*{Задание 7}
На $50$ сайтов одновременно совершают кибер-атаку $20$ хакеров. Каждый из них выбирает одну цель случайно и независимо от других, но всегда взламывает защиту. \par
Сколько сайтов в среднем останется не взломано?
\ssolve
Обозначим случайную величину $$X=\text{количество не взломанных сайтов}.$$ \par
Введём индикатор $$I_i = "\text{i-й сайт взломан}".$$
Пусть $I_i = 1$, если сайт взломан. Это событие случится если хотя бы один (допустим, первый) хакер из 20 выберет i-й сайт для взлома. Событие, при котором один хакер (с номером от 1 до 20) взломает i-й сайт, обозначим:
$$H_{i,1} = H_{i,2} = \ldots = H_{i,20},$$
при этом
 $$P(H_{i,1}) = P(H_{i,2}) = ... = P(H_{i,20}) = \frac{1}{50}.$$
Однако i-й сайт может одновременно выбрать $k \in [0, 20]$ хакеров, и любое из событий $H_{i, k}$ приведёт к событию «сайт взломан», т.е.
$$P(I_i=1) = H_{i,1} \cup H_{i,2} \cup \ldots \cup H_{i,20}.$$ \par
Поскольку события под знаком суммы независимы, воспользуемся теоремой о сумме вероятностей для подсчёта $P(I_i=1)$. Воспользуемся укороченной формулой для случая, когда случайные величины под знаком суммы независимы и одинаково распределены (это удовлетворяет нашему тривиальному случаю, где вероятности событий под знаком суммы равны) \\ ($\textit{en.wikipedia.org/wiki/Inclusion–exclusion\textunderscore principle\#In\textunderscore probability}$):

$$P(I_i=1) = P \left( \bigcup_{1}^{k=20}H_{i,k} \right) = 1 - (1-H_{i,k})^k = 1 - (1-\frac{1}{50})^{20} \simeq 0.332.$$
Далее, запишем
$$P(I_i = 1) = \E I_i \simeq 0.332,$$
что можно интерпретировать как "каждый сайт в среднем взломали 0,332 раза".

 \par 
Тогда
$$\E X = \E (50 - \sum_i \E I_i) = 50 - 50 \cdot 0.332 \simeq 33.380.$$  
\aanswer В среднем останется не взломано $\simeq 33.380$ 	сайтов.
\end{document}