\documentclass[a4paper,12pt]{article}
\usepackage[utf8]{inputenc}
\usepackage{amssymb}
\usepackage{amsmath}
\usepackage[T2A]{fontenc}
\usepackage[russian]{babel}
\pagenumbering{gobble}
\usepackage{multicol}
\newcommand{\ssolve}{\par\vspace{5pt}\noindent{\bf Решение. }\par}
\newcommand{\aanswer}{\par\vspace{5pt}\noindent{\bf Ответ. }}
\newcommand{\bpm}{\begin{pmatrix}}
\newcommand{\epm}{\end{pmatrix}}
\newcommand{\bvm}{\begin{vmatrix}}
\newcommand{\evm}{\end{vmatrix}}
\newcommand{\bal}{\left\{\begin{aligned}}
\newcommand{\eal}{\end{aligned}\right.}
\newcommand{\llaq}{\quad \longrightarrow \quad}
\newcommand{\ao}[1]{\overset{#1}{\longmapsto}}

\begin{document}

\title{<<Линейная алгебра:\\ Линейные отображения>>\\ \vspace{12pt} Домашнее задание №4}
\author{Байдаков Илья}
\date{\today}
\maketitle

\section*{Задание 1}
Найдите размерность и предъявите базис следующего подпространства в $\mathbb{R}^5$, заданного множеством решений следующей однородной системы линейных уравнений:

\[
\left\{
\begin{aligned}
& 2x_1 - x_2 + x_3 - 2x_4 + 4x_5 = 0\\
& 4x_1 - 2x_2 + 5x_3 + x_4 + 7x_5 = 0\\
& 2x_1 - x_2 + x_3 + 8x_4 + 2x_5 = 0
\end{aligned}
\right.
\]
\vspace{5pt}
\ssolve
Составим матрицу коэффициентов системы и приведём её к каноническому виду: \par
$$
A=
\begin{pmatrix}
2&-1&1&-2&4\\
4&-2&5&1&7\\
2&-1&1&8&2\\
\end{pmatrix}
\overset{III-I}{\longmapsto}
\begin{pmatrix}
2&-1&1&-2&4\\
4&-2&5&1&7\\
0&0&0&10&-2\\
\end{pmatrix}
\longmapsto
$$

$$
\overset{II-I \cdot 2}{\longmapsto}
\begin{pmatrix}
2&-1&1&-2&4\\
0&0&3&5&-1\\
0&0&0&10&-2\\
\end{pmatrix}
\overset{III \cdot \frac{1}{2}}{\longmapsto}
\begin{pmatrix}
2&-1&1&-2&4\\
0&0&3&5&-1\\
0&0&0&5&-1\\
\end{pmatrix}
$$ 

$$
\overset{I + II \cdot 4}{\longmapsto}
\begin{pmatrix}
2&-1&13&18&0\\
0&0&3&5&-1\\
0&0&0&5&-1\\
\end{pmatrix}
\overset{II-III}{\longmapsto}
\begin{pmatrix}
2&-1&13&18&0\\
0&0&1&0&0\\
0&0&0&5&-1\\
\end{pmatrix}
\longmapsto
$$

$$
\overset{I - III \cdot \frac{18}{5}}{\longmapsto}
\begin{pmatrix}
2&-1&13&18&0\\
0&0&1&0&0\\
0&0&0&5&-1\\
\end{pmatrix}
\overset{I+ III \cdot \frac{1}{2}}{\longmapsto}
\begin{pmatrix}
2&-1&13&0&\frac{18}{5}\\
0&0&1&0&0\\
0&0&0&5&-1\\
\end{pmatrix}
\longmapsto
$$

$$
\overset{I - II \cdot 13}{\longmapsto}
\begin{pmatrix}
2&-1&0&0&\frac{18}{5}\\
0&0&1&0&0\\
0&0&0&5&-1\\
\end{pmatrix}
$$

\par
Теперь видно, что размерность подпространства $V$, заданного множеством решений нашей однородной системы, равна количеству свободных переменных. У нас это $x_2$ и $x_5$:
$$\dim V = 2.$$ \par
Чтобы найти базис подпространства, будем попеременно задавать свободным переменным значения 1, остальных приравнивая к нулю.
$$
\left\{
\begin{aligned}
x_2=1 \\
x_5 = 0 \\
\end{aligned} 
\right.
\quad \longrightarrow \quad
v_2 = 
\begin{pmatrix}
\frac{1}{2} \\
1 \\
0 \\
0 \\
0 \\
\end{pmatrix}
$$

$$
\left\{
\begin{aligned}
x_2=0 \\
x_5 = 1 \\
\end{aligned} 
\right.
\quad \longrightarrow \quad
v_5 = 
\begin{pmatrix}
-\frac{9}{5} \\
0 \\
0 \\
- \frac{1}{5} \\
1 \\
\end{pmatrix}
$$
\aanswer
Размерность подпространства $\dim V = 2$, базис подпространства составлен векторами $v_2$ и $v_5$.


\section*{Задание 2}

Пусть $\varphi:\mathbb{R}^2\rightarrow \mathbb{R}^3$ --- линейное отображение, причём 
$$\varphi\begin{pmatrix}
{2}\\{1}
\end{pmatrix}=
\begin{pmatrix}
{2}\\{3}\\{2}
\end{pmatrix}, \
\varphi \begin{pmatrix}
{3}\\{0}
\end{pmatrix}=
\begin{pmatrix}
{1}\\{6}\\{9}
\end{pmatrix}.$$
Найдите $\varphi \begin{pmatrix}
-4 \\ 3
\end{pmatrix}$. \par
\vspace{8pt} 

\ssolve
Найдём матрицу линейного отображения $A$. Матрицы прообразов $V$ и образов $W$ записываются так:
$$
V=
\begin{pmatrix}
2&3\\
1&0\\
\end{pmatrix},
W=
\begin{pmatrix}
2&1\\
3&6\\
2&9\\
\end{pmatrix}
$$
Тогда матрица $A$ находится из выражения:
$$
A=WV^{-1}=
\begin{pmatrix}
2&1\\
3&6\\
2&9\\
\end{pmatrix}
\begin{pmatrix}
2&3\\
1&0\\
\end{pmatrix}^{-1}
=
\begin{pmatrix}
2&1\\
3&6\\
2&9\\
\end{pmatrix}
\begin{pmatrix}
0&1\\
\frac{1}{3}&-\frac{2}{3}\\
\end{pmatrix}
=
\begin{pmatrix}
\frac{1}{3}&1\frac{1}{3}\\
2&-1\\
3&-4\\
\end{pmatrix}
$$
Теперь подействуем матрицей $A$ на заданный вектор-прообраз из $V$:
$$\varphi \begin{pmatrix}
-4 \\ 3
\end{pmatrix}
=
A \cdot
\begin{pmatrix}
-4 \\ 3
\end{pmatrix} =
\begin{pmatrix}
2\frac{2}{3} \\ -11 \\ -24
\end{pmatrix}
$$
\aanswer
$\varphi 
\left( \begin{smallmatrix}
-4 \\ 3
\end{smallmatrix} \right)
=
\left( \begin{smallmatrix}
2\frac{2}{3} \\ -11 \\ -24
\end{smallmatrix} \right)
$.


\section*{Задание 3}

Пусть $V=\mathbb{R}[x]_{\leqslant 2}$ -- пространство многочленов степени не более 2. 
Линейный оператор $\varphi\colon V\rightarrow V$ в базисе $e = \{1,x,x^2\}$ задаётся матрицей
$$
A = A(\varphi, e) = \begin{pmatrix}
-1&3&2\\
-1&2&4\\
0&-1&2
\end{pmatrix}.
$$

а) Найдите $\varphi(x^2)$ и $\varphi(-2x^2 - 4x - 1)$.

\vspace{3pt}

б) Существует ли прообраз у многочлена $7x^2 + 2x - 5$? Иными словами, существует ли многочлен $g(x) \in V$ такой, что $$\varphi(g) = 7x^2 + 2x - 5?$$

\ssolve
а) В базисе $e = \{1,x,x^2\}$ многочлен $x^2$ соответствует вектору 
$$v_1=
\begin{pmatrix}
0 \\ 0 \\ 1
\end{pmatrix}
,$$
а значит, 		
$$\varphi(x^2)=A(\varphi, e)\cdot v_1 =
\begin{pmatrix}
-1&3&2\\
-1&2&4\\
0&-1&2
\end{pmatrix}
\begin{pmatrix}
0 \\ 0 \\ 1
\end{pmatrix}
=
\begin{pmatrix}
2 \\ 4 \\ 2
\end{pmatrix}
.$$ \par
Получаем, что искомый многочлен имеет вид:
$$w_1 = 2x^2+4x+2.$$

Аналогично, многочлен $-2x^2 - 4x - 1$ соответствует вектору 
$$v_2=
\begin{pmatrix}
-1 \\ -4 \\ -2
\end{pmatrix}
,$$
а значит, 		
$$\varphi(-2x^2 - 4x - 1)=A \cdot v_2 =
\begin{pmatrix}
-1&3&2\\
-1&2&4\\
0&-1&2
\end{pmatrix}
\begin{pmatrix}
-1 \\ -4 \\ -2
\end{pmatrix}
=
\begin{pmatrix}
-15 \\ -15 \\ 0
\end{pmatrix}.
$$ \par
Получаем, что искомый многочлен имеет вид:
$$w_2 = -15x-15.$$
\vspace{1pt} \par
б) Если прообраз $g$ у многочлена $w = 7x^2 + 2x - 5$ существует, то должно выполняться равенство:
$$A \cdot g = w 
\quad \longrightarrow \quad
A 
\begin{pmatrix}
g_1 \\ g_2 \\ g_3
\end{pmatrix}
 = 
\begin{pmatrix}
-5 \\ 2 \\ 7
\end{pmatrix}.$$
Получили систему линейных уравнений. Если решение удастся получить методом Гаусса, то $g(x)$ существует. Запишем расширенную матрицу системы и приведём её к каноническому виду:
$$ \left(\begin{array}{ccc|c}
-1&3&2&-5\\
-1&2&4&2\\
0&-1&2&7
\end{array} \right)
\longmapsto
\left(\begin{array}{ccc|c}
1&0&-8&-16\\
0&1&-2&-7\\
0&0&0&0
\end{array} \right).
$$
Принимая $g_1$, $g_2$ главными переменными, а $g_3$ --- зависимой, получаем решение:
$$
\left\{
\begin{aligned}
&g_1 = 8g_3-16\\
&g_2 = 2g_3-7 \\
&g_3 \in \mathbb{R}
\end{aligned}
\right.
\qquad \longrightarrow \qquad
g =
\begin{pmatrix}
8g_3-16 \\ 2g_3-7 \\ g_3
\end{pmatrix}, 
g_3 \in \mathbb{R}
$$ 
Вектор $g$ найден. Значит, прообраз существует.
\aanswer
а) $w_1 = 2x^2+4x+2, w_2 = -15x-15$. б) Существует.

\section*{Задание 4}
В пространстве $\mathbb R^3$ заданы следующие векторы:
\[
v_1 =
\begin{pmatrix}
{1}\\{0}\\{2}
\end{pmatrix}
,\;
v_2 =
\begin{pmatrix}
{1}\\{1}\\{-1}
\end{pmatrix}
,\;
v_3 =
\begin{pmatrix}
{2}\\{0}\\{3}
\end{pmatrix}
,\;
u_1 =
\begin{pmatrix}
{1}\\{0}\\{1}
\end{pmatrix}
,\;
u_2 =
\begin{pmatrix}
{0}\\{1}\\{1}
\end{pmatrix}
,\;
u_3 =
\begin{pmatrix}
{0}\\{0}\\{2}
\end{pmatrix}.
\]
Найдите матрицу $A$ линейного оператора $\varphi\colon \mathbb R^3\to \mathbb R^3$, заданного по правилу $x\mapsto Ax$, такого, что $A v_i = u_i$ при $i = 1, 2, 3$. 
\vspace{8pt}

\ssolve
Если матрица $A$ такого линейного оператора существует, должно выполняться равенство:
$$A \cdot V = U 
\quad \longrightarrow \quad
A \cdot 
\begin{pmatrix}
1&1&2\\
0&1&0\\
2&-1&3
\end{pmatrix}
=
\begin{pmatrix}
1&0&0\\
0&1&0\\
1&1&2
\end{pmatrix}
$$ \par
Это матричное уравнение можно решить домножением на $V^{-1}$ справа:
$$A  = U \cdot V^{-1} = 
U \cdot
\begin{pmatrix}
1&1&2\\
0&1&0\\
2&-1&3
\end{pmatrix}^{-1}
=$$

$$=
\begin{pmatrix}
1&0&0\\
0&1&0\\
1&1&2
\end{pmatrix}
\begin{pmatrix}
-3&5&2\\
0&1&0\\
2&-3&-1
\end{pmatrix}
=
\begin{pmatrix}
-3&5&2\\
0&1&0\\
1&0&0
\end{pmatrix}
$$ 
\aanswer
$A=\left(
\begin{smallmatrix}
-3&5&2\\
0&1&0\\
1&0&0
\end{smallmatrix} \right).$
\newpage
\section*{Задание 5}
В пространстве $\mathbb R^3$ заданы следующие векторы:
\[
v_1 = 
\begin{pmatrix}
{1}\\{1}\\{1}
\end{pmatrix},\,
v_2 = 
\begin{pmatrix}
{1}\\{2}\\{0}
\end{pmatrix},\,
v_3 = 
\begin{pmatrix}
{5}\\{8}\\{2}
\end{pmatrix},\,
v_4 = 
\begin{pmatrix}
{-1}\\{-2}\\{1}
\end{pmatrix},\,
v_5 = 
\begin{pmatrix}
{3}\\{5}\\{-1}
\end{pmatrix}.
\]
Существует ли линейное отображение $\varphi\colon \mathbb R^3\to \mathbb R^2$ такое, что $\varphi(v_i) = u_i$ при $i = 1, 2, 3, 4, 5$, где

\[
u_1 = 
\begin{pmatrix}
{1}\\{0}
\end{pmatrix},\,
u_2 = 
\begin{pmatrix}
{0}\\{1}
\end{pmatrix},\,
u_3 = 
\begin{pmatrix}
{2}\\{3}
\end{pmatrix},\,
u_4 = 
\begin{pmatrix}
{0}\\{0}
\end{pmatrix},\,
u_5 = 
\begin{pmatrix}
{1}\\{0}
\end{pmatrix}?
\]

\ssolve
Рассмотрим те из векторов-прообразов $v_i$, которые составляют базис $\mathbb{R}^3$: $v_2$, $v_3$ и $v_4$, и соответствующие им векторы-образы:
$$V=
\bpm
1&5&-1\\
2&8&-2\\
0&2&1
\epm, 
 U=
\bpm
0&2&0\\
1&3&0\\
\epm
$$

Если отображение $\varphi$ существует, то справедливо, что:
$$A(\varphi)V=U 
\quad \longrightarrow \quad
A(\varphi)= UV^{-1}=$$
$$
\bpm
0&2&0\\
1&3&0\\
\epm
\bpm
1&5&-1\\
2&8&-2\\
0&2&1
\epm^{-1}
=
\bpm
0&2&0\\
1&3&0\\
\epm
\bpm
-6&\frac{7}{2}&1\\
1&-\frac{1}{2}&0\\
-2&1&1
\epm=
\bpm
2&-1&0\\
-3&2&1\\
\epm
$$
Составим матрицу $V^*$ из всех данных нам векторов-прообразов и проверим, правильно ли матрица отображения работает с ними:
$$
U^*=A(\varphi)V^*=
\bpm
2&-1&0\\
-3&2&1\\
\epm
\bpm
1&1&5&-1&3\\
1&2&8&-2&5\\
1&0&2&1&-1
\epm
=
\bpm
1&0&2&0&1\\
0&1&3&0&0\\
\epm
$$
Столбцы матрицы $U^*$ соответствуют векторам $u_i$, значит, отображение существует.
\aanswer
Отображение существует.
\newpage

\section*{Задание 6}
Найти собственные значения и собственные векторы следующих матриц:

$$
\text{а)}  \begin{pmatrix}
2 & 8\\
3 & 4
\end{pmatrix}, \quad
\text{б)} \begin{pmatrix}
{4}&{-5}&{2}\\
{5}&{-7}&{3}\\
{6}&{-9}&{4}\\
\end{pmatrix}.
$$
Укажите базис собственных подпространств $V_\lambda$ для каждого собственного значения $\lambda$.

\ssolve
а) Составим характеристический многочлен $\chi_A(\lambda)$:
$$\chi_A(\lambda)=
\det(\lambda E - A)=
\bvm
\lambda - 2 & -8 \\
-3 & \lambda - 4
\evm
=$$
$$
=(\lambda - 4)(\lambda -2)-24= 
\lambda^2-6\lambda-16 
\llaq
\bal
&\lambda_1 = -2 \\
&\lambda_2 = 8
\eal
$$

Следуя алгоритму поиска базиса подпространств $V_\lambda$, найдём ФСР:
$$
A-\lambda_1 E = \bpm
4 & 8 \\
3 & 6 
\epm
\ao{\text{Элем. преобр.}}
\bpm
1 & 2 \\
0 & 0
\epm
\llaq 
v_1 = 
\bpm
-2 \\ 1
\epm
$$

$$
A-\lambda_2 E = \bpm
-6 & 8 \\
3 & -4 
\epm
\ao{\text{Элем. преобр.}}
\bpm
1 & -\frac{4}{3} \\
0 & 0
\epm
\llaq 
v_2 = 
\bpm
\frac{4}{3} \\ 1
\epm
$$
Значит, имеет два собственных значения матрицы, у каждого из которых есть подпространство, имеющее один базисный вектор.\par
\vspace{5pt}
б) Составим характеристический многочлен $\chi_A(\lambda)$:
$$\chi_A(\lambda)=
\det(\lambda E - A)=
\bvm
{\lambda - 4}&{5}&{-2}\\
{-5}&{\lambda +7}&{-3}\\
{-6}&{9}&{\lambda - 4}\\
\evm =$$
$$
=(\lambda-4)^2(\lambda+7)+90+90-12(\lambda+7)+25(\lambda-4)+27(\lambda-4)=
$$
$$=\lambda^3-\lambda^2=\lambda^2(\lambda-1) 
\llaq
\bal
&\lambda_1 = 0 \\
&\lambda_2 = 1
\eal
$$
Теперь найдём ФСР:
$$
A-\lambda_1 E = A 
\ao{\text{Элем. преобр.}}
\bpm
{1}&{0}&{-\frac{1}{3}}\\
{0}&{1}&{-\frac{2}{3}}\\
{0}&{0}&{0}\\
\epm
\llaq 
v_1 = 
\bpm
\frac{1}{3} \\ \frac{2}{3} \\ 1
\epm
$$
$$
A- \lambda_2 E = 
\bpm
{3}&{-5}&{2}\\
{5}&{-8}&{3}\\
{6}&{-9}&{3}\\
\epm
\ao{\text{Элем. преобр.}}
\bpm
{1}&{0}&{-1}\\
{0}&{1}&{-1}\\
{0}&{0}&{0}\\
\epm
\llaq 
v_2 = 
\bpm
1 \\ 1 \\ 1
\epm
$$
\aanswer \par
\begin{multicols}{2}
а) Собственные значения: \\ $\lambda_1 = -2$, $\lambda_2 = 8$, \\
базисы подпространств: \\ 
$V_{\lambda1}=\langle \{v_1\} \rangle$,
$v_1 = \bpm -2 \\ 1 \epm$, \\
$V_{\lambda2}=\langle \{v_2\} \rangle$,
$v_2 = \bpm \frac{4}{3} \\ 1 \epm$, \\
собственные векторы: \\
$v_{\lambda1}=x\cdot v_{\lambda1}, \\
v_{\lambda2}=x\cdot v_{\lambda2}, x \in \mathbb{R}$

\columnbreak
б) Собственные значения: \\ $\lambda_1 = 0$, $\lambda_2 = 1$, \\
базисы подпространств: \\ 
$V_{\lambda1}=\langle \{v_1\} \rangle$,
$v_1 = \bpm \frac{1}{3} \\ \frac{2}{3} \\ 1 \epm$, \\
$V_{\lambda2}=\langle \{v_2\} \rangle$,
$v_2 = \bpm 1 \\ 1 \\ 1 \epm$, \\
собственные векторы: \\
$v_{\lambda1}=x\cdot v_{\lambda1}, \\
v_{\lambda2}=x\cdot v_{\lambda2}, x \in \mathbb{R}$
\end{multicols}


\end{document}