\documentclass[a4paper,12pt]{article}
\usepackage[utf8]{inputenc}
\usepackage{graphicx}
\usepackage{amssymb}
\usepackage{amsmath}
\usepackage[T2A]{fontenc}
\usepackage[russian]{babel}
\pagenumbering{gobble}


\begin{document}

\title{<<Дискретная математика: \\ комбинаторика и вероятность>> \\ \vspace{12pt} Домашнее задание}
\author{Байдаков Илья}
\date{\today}
\maketitle


\section*{Задание 1}
После опроса $250$ человек оказалось, что английский знают ровно $210$ респондентов, испанский --- $100$, а оба языка --- $80$. Сколько из опрошенных не знают ни английского, ни испанского? \par

\noindent {\bf Решение.} Обозначим различные множества, данные условием задачи:
\begin{itemize}
\item Все опрошенные --- $L$, при этом $\vert L \vert = 250$
\item Знающие английский --- $E$, при этом $\vert E \vert = 210$
\item Знающие испанский --- $S$, при этом $\vert S \vert = 100$
\end{itemize}
Количество знающих одновременно испанский и английский:
$$ \vert E \cap S \vert = 80 $$
Количество знающих английский и/или испанский:
$$ \vert E \cup S \vert $$
Количество не знающих английского и испанского, т.е. искомое число:
$$ \vert L \setminus ( E \cup S )\vert $$
Согласно правилу суммы,
$$ \vert E \cup S \vert = \vert E \vert + \vert S \vert - \vert E \cap S \vert = 210 + 100 - 80 = 230 $$
Искомое число вычисляется как разность множеств: 
$$ \vert L \setminus ( E \cup S )\vert =250-230=20$$
{\bf Ответ.} 20 опрошенных.

\section*{Задание 2}
Есть $10$ кандидатов на $6$ различных вакансий. Каждого кандидата можно взять на любую вакансию. Сколькими способами можно заполнить вакансии? (Каждая вакансия должна быть заполнена ровно одним человеком.)\par 
{\noindent \bf Решение.} В данном случае требуется узнать, сколько способов существует для выбора упорядоченного набора 6 элементов из 10-элементного множества, т.е. число размещений из 10 по 6:
$$A_{10}^6=\frac{10!}{(10-6)!}=151200$$
{\bf Ответ.} 151200 способами.

\section*{Задание 3}
Найдите вероятность того, что в случайном шестизначном коде будет хотя бы две одинаковые цифры.\par
{\noindent \bf Решение.} Количество возможных шестизначных кодов (кардиналь-\\ность пространства элементарных исходов):
$$\vert\Omega|=10^6$$
Пусть $A_2$ --- событие <<хотя бы две одинаковые цифры>>. \\
Рассмотрим $\neg A_2=$<<все цифры по одному разу>>. Ему соответствует множество $\Omega\setminus A_2$. \\ Количество 6-значных кодов, в которых <<все цифры по одному разу>>, при количестве цифр 10, соответствует числу размещений из 10 по 6, т.е. ответу предыдущей задачи: 
$$|\Omega\setminus A_2|=151200$$
По правилу суммы, 
$$|A_2| = |\Omega| - |\Omega\setminus A_2| = 10^6 - 151200 = 848800$$
Искомая вероятность, по определению, равна:
$$P(A_2) = |A_2|/|\Omega|=0,8488$$
{\bf Ответ.} $P = 0,8488$.

\section*{Задание 4}
{\bf a)} Каких натуральных чисел больше среди первого миллиона: тех, в записи которых
есть единица или тех, в записи которых её нет?\\
{\bf б)} Тот же вопрос для первых 10 миллионов чисел.\par
{\noindent \bf Решение.}
Найдём количество $a^{million}$ чисел среди первого миллиона натуральных чисел, в записи которых есть единица. Первый миллион натуральных чисел записывается как последовательность:
$$0, 1, \ldots , 999998, 999999$$
По её виду можно утверждать, что:
$$a^{million} = a_1 + a_2 + a_3 + a_4 + a_5 + a_6,\qquad (*)$$ 
где $a_x$ - число $x$-значных чисел, в записи которых есть единица. \\
Очевидно, $a_1 = 1$.\\
Далее, $$a_2 = 10 + 8 = 18,$$ т.е. $10$ цифр от $10$ до $19$ и $8$ цифр в каждом из $8$ десятков ($21, 31\ldots91$).\\
Далее, $$a_3 = 100 + a_1\cdot8+a_2\cdot8=252,$$ т.е. $100$ цифр от $100$ до $199$, а также $8$ цифр $201, 301\ldots901$, а также $8$ раз по $a_2$. Последнее слагаемое учитывает все трёхзначные числа, в записи которых есть единица, но без нуля во втором разряде (они учтены во втором слагаемом) и без единицы в третьем разряде (они в первом слагаемом). Продолжая далее до $a_6$ и вынося $8$ за скобки, видно, что формулу для $a_x$ можно записать в виде рекуррентной формулы:
\begin{equation}
  a_x=\begin{cases}
    1 & \text{если $x=1$},\\
    \displaystyle\sum_{k=1}^{x-1}{a_k}\cdot8+10^{x-1} & \text{если $x>1$}.
  \end{cases}
\end{equation}
Согласно (*),
$$a^{million} = \sum_{k=1}^{6}{a_k}=468559$$
Это меньше половины миллиона, значит, среди первого миллиона натуральных чисел больше тех, в записи которых нет единицы (вопрос \textbf{\textit{а}}).
Для вопроса \textbf{\textit{б}}:
$$a^{10-million} = \sum_{k=1}^{7}{a_k}=5217031>5000000,$$
т.е. ответ противоположный.\par
{\bf Ответ.} а --- больше без единицы, б --- больше с единицей. \par

Решение через правило произведения ($9^6>10^6/2$ и $9^7<10^7/2$) пришло в голову уже после оформления вышеприведённого.

\section*{Задание 5}
Найдите вероятность дубля при броске двух кубиков (дубль означает, что на обоих кубиках выпало одинаковое значение).\par
{\noindent \bf Решение.} Количество возможных исходов при броске двух кубиков, по правилу произведения:
$$|\Omega| = 6^2.$$
Пусть событие $A_2$ --- выпадение дубля, т.е. $A_2=\{11,22,33,44,55,66\}$.\\
Вероятность выпадения дубля:
$$P(A_2)=\frac{|A_2|}{|\Omega|}=\frac{1}{6}.$$
{\bf Ответ.}  $P=1/6$.\par

\section*{Задание 6}
Команда принимает участие в турнире, где сыграет {\it четыре} игры.\\
Вероятность выиграть в первом матче равна $1/2$. Вероятность выигрыша после победы в предыдущем матче возрастает до $2/3$, а после поражения уменьшается до $1/3$. Какова вероятность\\
{\bf a)} выиграть не менее двух игр?\\
{\bf б)} выиграть ровно две игры?\par
{\noindent \bf Решение.} Для ответа на вопрос {\bf(а)} рассмотрим отрицательное событие к событию $W_{2+}=$<<выиграть не менее двух игр>>, т.е. $\neg W_{2+}=$<<проиграть три или четыре игры>>:
$$\neg W_{2+}=\{\textit{ПППВ, ППВП, ПВПП, ВППП, ПППП}\}.$$
Найдём вероятность такого события. Для этого посчитаем вероятность каждого из пяти исходов в $\neg W_{2+}$ и найдём их сумму. \par
Для каждого из случаев \{\textit{ПППВ,ВППП}\} вероятность будет произведением множителей $1/2,2/3,2/3,1/3$, т.е. $2/27$.
Для каждого из случаев \{\textit{ППВП,ПВПП}\} вероятность будет произведением множителей $1/2,2/3$, $1/3,1/3$, т.е. $1/27$.
Для случая \{\textit{ПППП}\} вероятность будет произведением множителей $1/2,2/3$,$2/3,2/3$, т.е. $4/27$.
Сумма вероятностей этих пяти исходов есть вероятность события:
$$P(\neg W_{2+})=2/27\cdot2+1/27\cdot2+4/27=10/27.$$
Значит, вероятность искомого события $W_{2+}$ --- вероятность отрицания к событию $\neg W_{2+}$:
$$P(W_{2+})=1-10/27=17/27.$$
Для ответа на вопрос {\bf(б)} будем использовать аналогичный подход. Рассмотрим отрицательное событие к событию $W_{2}=$<<выиграть две игры>>, т.е. $\neg W_{2}=$<<проиграть все игры, выиграть все игры, проиграть одну игру, выиграть одну игру>>. \par По условию задачи вероятности <<выиграть>> или <<проиграть>> численно равны при симметрично равных исходах остальных игр, значит, первые два исхода в $\neg W_{2}$ имеют одинаковую вероятность, так же, как и два последних. Они уже рассчитаны в пункте {\bf(а)}. Сложим их:
$$P(\neg W_{2})=4/27\cdot2+(1/27+2/27)\cdot2\cdot2=20/27.$$
Аналогично, вероятность искомого события $W_{2}$:
$$P(W_{2})=1-20/27=7/27.$$
{\bf Ответ.} а --- $17/27$, б --- $7/27$. \par

\section*{Задание 7}
Монету бросают восемь раз. Найдите вероятности событий:

{\bf a)} $A$ --- "орел выпал 6 раз";

%{\bf б)} $B$ --- "орел выпал не более двух раз";

{\bf б)} $B$ --- "орел выпал не менее трех раз".

{\noindent \bf Решение.} Количество возможных исходов эксперимента при броске монеты восемь раз, по правилу произведения:
$$|\Omega| = 2^8.$$
Количество возможных исходов, соответствующих событию <<орёл выпал 6 раз>> соответствует числу сочетаний из $8$ по $2$, т.е. 
$$|A|=C_8^2=\frac{8!}{2!(8-2)!}=28.$$
Вероятность события, по определению:
$$P(A)=\frac{|A|}{|\Omega|}=\frac{28}{2^8}=\frac{7}{64}.$$
Для ответа на вопрос {\bf(б)} рассмотрим событие, отрицательное к событию $B=$<<орёл выпал не менее трёх раз>>, т.е. $\neg B=$<<орёл выпал один или два раза>>.\par
Для <<орёл выпал два раза>> количество возможных исходов численно равно рассчитанному в пункте {\bf(а)}, т.е. $28$.\par 
Для <<орёл выпал один раз>> количество возможных исходов рассчитывается аналогично как число сочетаний из $8$ по $1$ и равно $8$. \par
Сразу рассчитаем вероятность отрицания к событию $\neg B=$<<орёл выпал один или два раза>>, суммируя количество возможных исходов:
$$P(B)=1-\frac{28+8}{2^8}=\frac{55}{64}.$$ 
{\bf Ответ.} $P(A)=7/64$, $P(B)=55/64$. \par
\end{document}