\documentclass[a4paper,12pt]{article}
\usepackage[utf8]{inputenc}
\usepackage{graphicx}
\usepackage{amssymb}
\usepackage{amsmath}
\usepackage[T2A]{fontenc}
\usepackage[russian]{babel}
\pagenumbering{gobble}
\usepackage{multicol}
\newcommand{\ssolve}{\par\vspace{5pt}\noindent{\bf Решение. }\par}
\newcommand{\aanswer}{\par\vspace{5pt}\noindent{\bf Ответ. }}
\newcommand{\bpm}{\begin{pmatrix}}
\newcommand{\epm}{\end{pmatrix}}
\newcommand{\bvm}{\begin{vmatrix}}
\newcommand{\evm}{\end{vmatrix}}
\newcommand{\bal}{\left\{\begin{aligned}}
\newcommand{\eal}{\end{aligned}\right.}
\newcommand{\llaq}{\quad \longrightarrow \quad}
\newcommand{\ao}[1]{\overset{#1}{\longmapsto}}

\begin{document}

\title{<<Теория вероятностей:\\ Условная вероятность>>\\ \vspace{12pt} Домашнее задание №3}
\author{Байдаков Илья}
\date{\today}
\maketitle

\section*{Задание 1}
В коробке лежат $100$ карточек с числами $00, 01, 02, \ldots, 98, 99$. Вася достаёт одну карточку из коробки, считает сумму $S_1$ и произведение $S_2$ цифр на ней. Найдите вероятность $P(S_1 = i|S_2 = 0)$, где $i = 0, 1, \ldots , 18$.
\ssolve
Запишем, что $$P(S_2) = \frac{\text{кол-во карточек с 0 в номере}}{100} = \frac{19}{100},$$
а также, имея ввиду карточку $00$,
$$P(S_1=0 \cap S_2 = 0) = \frac{\text{кол-во карточек c 0 и суммой 0}}{100} = \frac{1}{100}.$$
Имея ввиду карточки $01$ и $10$ (аналогично для пар карточек $02, 20$ при $S_1=2$ т.д.):
$$P(S_1=1 \cap S_2 = 0) = \frac{\text{кол-во карточек c 0 и с суммой 1}}{100} = \frac{2}{100}$$
Также заметим, что карточек с нулём в номере и с суммой больше 9 нет в коробке, поэтому для них
$$P(S_1>9 \cap S_2=0) = 0.$$
Тогда, используя определение условной вероятности:
$$P(S_1=i | S_2 = 0) = \frac{P(S_1=i \cap S_2 = 0)}{P(S_2)},$$
получаем \textbf{ответ:} \\
$P(S_1=i | S_2 = 0) = \frac{1}{19} \text{ при } i = 0 $ \\
$P(S_1=i | S_2 = 0) = \frac{2}{19} \text{ при } i = 1, 2, \ldots, 9,$ \\
$P(S_1=i | S_2 = 0) = 0 \text{ при } i = 10, 11, \ldots, 18.$
 
\section*{Задание 2}
Даны две урны. В первой лежат $1$ белый и $9$ черных шаров, а во второй — $5$ белых и $1$ черный. Из каждой урны достали по одному шару без возвращения. Оставшиеся в двух урнах шары ссыпали в третью урну. Какова вероятность того, что шар, вытянутый из третьей урны, окажется черным?
\ssolve
Заметим, что достать по одному шару из каждой урны можно одним из четырёх способов: \par
I -- из обеих урн по чёрному шару, \par
II(а) -- из первой белый шар, из второй чёрный шар,\par
II(б) -- из первой чёрный шар, из второй белый шар,\par
III -- из обеих урн по белому шару.\par
При этом II(а) и II(б) равнозначно влияют на вероятность, с которой из третьей урны достанут чёрный шар, и их можно объединить в одно событие II. Тогда события I, II и III создают разбиение пространства $\Omega$. \par
Найдём вероятности $P(I)$, $P(II)$ и $P(III)$. Учитывая, что события \\ $"\text{достать шар определённого цвета из первой урны}"$ и то же \\ $"\text{... из второй урны}"$  независимы, а вероятность каждого из них есть отношение шаров определённого цвета к общему кол-ву шаров в урне, находим:
$$P(I) = \frac{9}{10}\cdot \frac{1}{6} = \frac{9}{60}$$
$$P(II) =\frac{1}{10}\cdot \frac{1}{6}+\frac{9}{10}\cdot \frac{5}{6} = \frac{46}{60}$$ 
$$P(III) = \frac{1}{10}\cdot \frac{5}{6} = \frac{5}{60}$$
Обозначим событие $A= "\text{достать чёрный шар из третьей урны}"$. Учтём, что вероятность достать чёрный шар после событий $I$, $II$ и $III$ равна отношению чёрных шаров в третьей урне к общему количеству в ней. \par Используя формулу полной вероятности, найдём ответ:
$$P(A)=P(A|I)\cdot P(I)+P(A|II)\cdot P(II)+P(A|III)\cdot P(III)=$$
$$=\frac{8}{14}\cdot \frac{9}{60}+\frac{9}{14}\cdot \frac{46}{60}+\frac{10}{14}\cdot \frac{5}{60}\simeq 0.64$$
\aanswer $P \simeq 0.64$.

\section*{Задание 3}
Три студента пишут контрольную работу из 4-х задач. Первый студент решает любую задачу с вероятностью $\frac{3}{4}$, второй — с вероятностью $\frac{1}{2}$, третий — $\frac{1}{4}$. Преподаватель получил анонимную работу с тремя решенными задачами. Кому скорее всего принадлежит работа? Найдите вероятность, с которой работа принадлежит этому студенту.

\ssolve
Обозначим событие $A_i = "\text{работа принадлежит $i$-му студенту}"$ (без информации о решённых задачах), т.е.
$$P(A_1)=P(A_2)=P(A_3)=\frac{1}{3}.$$
Обозначим событие $B = "\text{в работе решены три задачи}"$, тогда:
$$P(B|A_1)=(\frac{3}{4})^3 \cdot \frac{1}{4} \simeq 0.105$$
$$P(B|A_2)=(\frac{1}{2})^3 \cdot \frac{1}{2} \simeq 0.063$$
$$P(B|A_3)=(\frac{1}{4})^3 \cdot \frac{3}{4} \simeq 0.012.$$
Значит, работа скорее принадлежит первому студенту, поскольку он с большей вероятностью, чем остальные студенты, решает три задачи из четырёх. \par
Вероятность, что работа принадлежит ему, есть условная вероятность $P(A_1|B)="\text{работа принадлежит $1$-му ст-ту, если решены три задачи}"$. 
По формуле Байеса:
$$P(A_1|B)=\frac{P(B|A_1)P(A_1)}{P(B)}. \qquad \qquad (*)$$
Поскольку $A_1$, $A_2$ и $A_3$ создают разбиение, запишем формулу полной вероятности:
$$P(B)=P(A_1)P(B|A_1)+P(A_2)P(B|A_2)+P(A_3)P(B|A_3)=$$
$$=\frac{1}{3}(0.105+0.063+0.012) \simeq 0.0599$$
Тогда по формуле (*) находим:
$$P(A_1|B)=\frac{P(B|A_1)P(A_1)}{P(B)}=\frac{0.105\cdot\frac{1}{3}}{0.0599}\simeq 0.59.$$
\aanswer Скорее всего первому студенту. Вероятность этого $P \simeq 0.59$.
\section*{Задание 4}
Для оценки числа некоторого редкого вида рыб в озере биологи выловили $5$ рыб и
пометили их. На следующий день они выловили $2$ рыбы. Пусть случайная величина $X$ —
число помеченных рыб среди выловленных.
При каком количестве $N$ рыб в озере вероятность $P (X = 1)$ максимальна? Найдите
распределение случайной величины $X$ при таком $N$.
\ssolve
По определению, 
$$P (X=1)= \frac{"\text{число способов выловить 2 рыбы, в т.ч. 1 помеченую}"}{C_N^2}=$$
$$=\frac{5 (N-5)}{\frac{N!}{2!(N-2!)}}=\frac{5(N-5)\cdot 2! (N-2)!}{N!} = \frac{10(N-5)}{N(N-1)}$$ 
Чтобы найти максимум $P(X=1)(N)$, исследуем полученное выражение на экстремум:
$$\left(\frac{10(N-5)}{N(N-1)}\right)'=\frac{10N^2+100N-50}{N(N-1)}$$
Приравнивая числитель к нулю и решая квадратное уравнение, получаем, что
$N=5+2\sqrt{5}\simeq 9.47$ или $N= 5-2\sqrt{5}\simeq 0.52$. Поскольку $N>5$, подходит первый вариант. Проверяя ближайшие целые числа $N=9$ и $N=10$ полученной выше формулой для $P(X=1)$, находим, что  
$$P(X=1)_{N=9}\quad = \quad P(X=1)_{N=10} \quad =\quad0.5(5).$$
Чтобы найти распределение $X$, будем искать каждую вероятность по определению, отдельно для $N=9$ и $N=10$. Например, для $N=9$ значение $P(X=2)$ находится так:
$$P(X=2) = \frac{C_5^2}{C_9^2}=\frac{10}{36}=0.27(7).$$
Получаем такие распределения:
\par \vspace{5pt}
Для $N=9$:
\begin{tabular}{|c|c|c|c|}
\hline
$X$&0&1&2 \\
\hline
$P$&0.16(6)&0.5(5)&0.27(7)\\
\hline
\end{tabular}
\par \vspace{5pt}
Для $N=10$:
\begin{tabular}{|c|c|c|c|}
\hline
$X$&0&1&2 \\
\hline
$P$&0.2(2)&0.5(5)&0.2(2)\\
\hline
\end{tabular} \par
\aanswer $N=9, N=10$. Распределения приведены в таблицах выше.
\section*{Задание 5}
Совместное распределение случайных величин
задано таблицей.\par
\includegraphics[width=0.7 \textwidth]{2022-03-12_011905.png}\par
а) Какова вероятность того, что $X=\frac{1}{2}$, а $Y=1$? \par
б) Найдите $P(X=-1)$ \par
в) Найдите $P (Y = 1)$\par
г) Найдите $P (X^2 + Y = 2)$.\par
\aanswer \par
а) $P(X=\frac{1}{2},Y=1) = 0.1$
б) $P(X=-1)=0.15$ \par
в) $P (Y = 1)=0.35$\par
г) $P (X^2 + Y = 2)=0.05$, т.к. это возможно только при $X=-1, Y=1$.\par
\section*{Задание 6}
Совместное распределение двух непрерывных случайных величин $X$ и $Y$ задано плотностью $f_{X,Y}(x, y) = \frac{1}{4}e^{-|x|-|y|}$. \par
Верно ли, что случайные величины X и Y независимы?
\aanswer \par
Верно, потому что существует разложение $f_{X,Y}(x, y)=f_X(x)\cdot f_Y(y)=\frac{1}{2}e^{-|x|}\cdot \frac{1}{2}e^{-|y|}$, при этом $f_X(x)$ и $f_Y(y)$ удовлетворяют свойствам плотности случайной величины.
\section*{Задание 7}
Случайная величина $X$ имеет плотность $f_X(x) = \frac{1}{\pi(x^2+1)}$, а $Y$ имеет плотность
$f_Y (x) = \frac{1}{\sqrt{2 \pi}}e^{-\frac{x^2}{2}}$.
Какова плотность совместного распределения X и Y , если известно, что они независимы?\par
\aanswer \par
Плотность совместного распределения является произведением данных плотностей: \par
$$f_{X,Y}(x_1, x_2) =  \frac{1}{\pi(x_1^2+1)} \cdot \frac{1}{\sqrt{2 \pi}}e^{-\frac{x_2^2}{2}}.$$ \par
\section*{Задание 8}
Функция распределения случайного вектора $(X, Y )$ имеет вид
$$F_{X,Y} (x, y) = \frac{1}{\pi^2}\arctg(x)\cdot \arctg(y) + \frac{1}{2\pi}\arctg(x)+\frac{1}{2\pi}\arctg(y)+\frac{1}{4}$$ \par
Чему равняется вероятность $P (1 < X \leq \sqrt{3}, 0 < Y \leq 1)$?
\ssolve
$$P (1 < X \leq \sqrt{3}, 0 < Y \leq 1) = F_{X,Y} (\sqrt{3}, 1) -  F_{X,Y} (1, 0) = $$
$$\frac{1}{\pi^2}\arctg(\sqrt{3})\cdot \arctg(1) + \frac{1}{2\pi}\arctg(\sqrt{3})+\frac{1}{2\pi}\arctg(1)+\frac{1}{4} - $$
$$-\frac{1}{\pi^2}\arctg(1)\cdot \arctg(0) - \frac{1}{2\pi}\arctg(1)-\frac{1}{2\pi}\arctg(0)-\frac{1}{4}=$$
$$=\frac{1}{\pi^2}\arctg(\sqrt{3})\cdot \arctg(1) + \frac{1}{2\pi}\arctg(\sqrt{3})-\frac{1}{\pi^2}\arctg(1)\cdot \arctg(0)-\frac{1}{2\pi}\arctg(0)=$$
$$=\frac{1}{\pi^2} \arctg(1)(\arctg(\sqrt{3})-\arctg(0)) + \frac{1}{2\pi}(\arctg(\sqrt{3})-\arctg(0)) =$$
$$=(\arctg(\sqrt{3})-\arctg(0))(\frac{1}{\pi^2} \arctg(1)+\frac{1}{2\pi})=$$
$$=(\frac{\pi}{3}-0)(\frac{1}{\pi^2}\cdot\frac{\pi}{4}+\frac{1}{2\pi})=\frac{\pi^2}{12\pi^2}+\frac{\pi}{6\pi}=\frac{1}{12}+\frac{1}{6}=\frac{1}{4}$$
\aanswer $P=\frac{1}{4}.$
\end{document}